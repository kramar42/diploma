\newpage

\selectlanguage{ukrainian}
\section{Вивід системи рівнянь для опису системи}



\hspace*{8mm} Основне рівняння, що описує еволюцію світла в подібних середовищах - нелінійне рівняння Шредінгера. Почнемо його вивід з рівнянь Максвела. З них можна отримати наступне хвильове рівняння (в параксіальному наближенні) для напруженості електричного поля пов'язаного з світловим пучком:
\begin{equation}\label{eq:waveeq}
\nabla^2 {\bf E} - \dfrac{1}{c^2}\frac{\partial^2 {\bf E}}{\partial^2 t} = \dfrac{1}{\epsilon_0 c^2}\frac{\partial^2 {\bf P}}{\partial^2 t}
\end{equation}
де $c$ - швидкість світла у вакуумі, $\epsilon_0$ - проникливість вакууму, ${\bf P}$ - поляризація.

Ми розглядаємо випадок, коли хвиля розпосюджується по осі Z, а розсіюється та самофокусується по осях X та Y. До того ж, ми вважаємо світло квазімонохроматичним. В наближенні повільно змінюваної обвідної корисно було б виділити швидко змінювану частину:

\begin{equation}\label{eq:qseq}
{\bf E} = \dfrac{1}{2}[E({\bf r},t)exp(-i\omega_0 t) + c.c.]
\end{equation}
та\\

\begin{equation}\label{eq:qseq2}
E({\bf r},t) = A({\bf r})exp(-i\beta_0 Z)
\end{equation}

де $\beta_0 = k_0 n_0$ - постійна розповсюдження.
Вважаємо що ${\bf P} = n_{nl} (|{\bf E}|^2){\bf E}$. Крім того, через те, що обідна повільно змінюється по осі Z, ми можемо знехтувати доданком $\frac{\partial^2 A}{\partial^2 Z}$ та врешті-решт отримаємо:


\begin{equation}\label{eq:almNLS}
2i\beta_0 \frac{\partial A}{\partial Z} + \frac{\partial^2 A}{\partial^2 X} + \frac{\partial^2 A}{\partial^2 Y} + 2\beta_0 k_0 nl(I)A = 0
\end{equation}

або, після обезрозмірення:

\begin{equation}\label{eq:NLS1} 
i\partial_z \Psi+\Delta_\perp\Psi +\theta\Psi=0
\end{equation}

Коефіцієнт заломлення, в свою чергу рахується таким чином:

\begin{equation}
\alpha^2\,\theta-\Delta_\perp\theta=I
\end{equation}

де $I$ - інтенсивність світла, $\alpha$ - параметр нелокальності. Як бачимо, коефіцієнт для заломлення залежить не лише від інтенсивності в даній точці, але від інтенсивності у всіх точках перерізу.

Для двохкомпонентного пучку (векторного вихору) отримуємо наступну систему:
\begin{equation}
   \begin{array}{l} {\displaystyle
       i\partial_z \Psi_n+\Delta_\perp\Psi_n +\theta\Psi_n=0,
       } \\*[9pt] {\displaystyle
\alpha^2\,\theta-\Delta_\perp\theta=|\Psi_1|^2+|\Psi_2|^2.
   }\end{array}
   \label{eq:NLS}
\end{equation}
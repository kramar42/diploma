\newpage
\section{Вступ}
Го --- стародавня китайська стратегічна гра на двох гравців.
Вона є антагоністичною\footnote{Антагоністичні ігри —-- ігри з двома гравцями які мають прямо протилежні інтереси} грою з повною інформацією\footnote{Гра з повною інформацією --- термін теорії ігор, що позначає логічну гру, в якій для суперників відсутній елемент невизначеності}.
В го грають два гравці --- ``Чорні'' та ``Білі''. Вони по черзі розміщують на дошці, що складається з перетину 19 на 19 ліній, камені свого кольору.
Го територіальна гра, тобто гравець, що під кінець гри має більшу територію --- виграє.
Як і для багатьох інших подібних ігор, були спроби створити комп'ютерні програми, що гарно грають в го, але це виявилося справжнім викликом для програмістів.
Складність обчислення партій в го на кілька порядків більша за шахи.
На кожному кроці можливі близько 200—300 ходів, статична ж оцінка життя груп каменів фактично неможлива.
Одним ходом тут можна цілком зіпсувати всю гру, навіть коли решта ходів були дуже добрі.
Тому програми для гри в го не використовують таких алгоритмів, як шахові програми, а замість того зазвичай мають кілька десятків модулів для оцінки різних аспектів гри і під час аналізу намагаються використовувати ті ж самі поняттями, що й люди.
Попри це вони і далі грають дуже слабко та програють навіть не дуже сильним аматорам.

Перша програма для гри в Го була написана Альбертом Зобріст в 1968 році як частина дисертації по розпізнаванню образів. Він використав функцію впливу для оцінкі території і Зобріст-хешування для виявлення ситуацій Ко.

Останні розробки в пошуку Монте-Карло по деревах і машинному навчанню принесли кращі результати для програм, що грають в го на маленькій дошці 9x9. У 2009 році з'явилися перші подібні програми, що могли змагатися з низькими данами на дошках розміром 19х19.

Усі програми, що грають в го повинні оперувати деякими структурами даних (наприклад такими, що репрезентують поточний стан дошки або гри).
Більшість з таких програм засновані на створенні великої кількості різних партій, та подальшому їх аналізі.
Саме тому питання пошуку у подібних структурах даних дуже актуальне, адже його оптимізація корінним чином вплине на роботу та якість програм, що грають в го.
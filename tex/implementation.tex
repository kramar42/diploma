\newpage
\section{Вибір засобів реалізації}
У попередньому розділі було прийнято рішення розроблювати бібліотеку, що буде дозволяти шукати у структурах даних гри го декількома різними способами. Ця бібліотеку може бути використана програмами, що аналізують партії в го, збирають статистику з партій або ж шукають деякі закономірності у партіях. Щоб ця бібліотека була корисною і використовувалася, важливо вибрати правильну платформу та мову програмування для неї.

Основними властивостями розроблювальної бібліотеки повинні бути:
\begin{itemize}
	\item Платформонезалежність --- це зробить бібліотеку більш поширеною та зручною.
	\item Можливість використання у багатьох популярних мовах програмування (загалом, це доповнення до попереднього пункту).
	\item Простота реалізації. Бажано, щоб внутрішні структури даних, були змодельовані використовуючи вбудовані структури даних мови.
	\item Бажано, щоб мова підтримувала функціональний підхід програмування. Функціональний підхід гарно себе зарекомендував при роботі з штучним інтелектом або якимось складним аналізом даних.
\end{itemize}

Розглянемо деякі можливі платформи для поставленої задачі.
\subsection{Вибір платформи}
Вибір платформи для програми, це дуже важливе рішення, адже платформа одразу не тільки поставить обмеження на доступні мови програмування але і додасть свої плюси та мінуси, до розроблюваного програмного забезпечення. Загалом, дуже загально, можна разбити платформи на три пункти:
\begin{itemize}
	\item Без використання платформи
	\item .NET
	\item Java
\end{itemize}

Розглянемо їх по черзі оцінюючи доступні мови програмування для кожної платформи та їх плюси і мінуси.
\subsubsection{Без платформи}
Найочевиднішим вибором буде не використовувати ніяку платформу. Адже навіщо ускладнювати, якщо можна зробити простіше. Однак програмування, використовуючи платформу, таку як Java або .NET, має свої значні переваги, серед яких більшість плюсів надається єко-системою платформи. Однак при програмування бібліотеки методів пошуку, не обов'язково використовувати якусь платформу, тож розглянемо цей варіант.

Основними перевагами написання програми без використання будь-якої платформи є:
\begin{itemize}
	\item Швидкодія (не завжди)
	\item Відсутність прошарків між програмою та ОС (найчастіше)
	\item Не прив'язаність ні до яких інструментів
	\item Можливість написати простий standalone-додаток
\end{itemize}

Для бібліотеки важливими можуть вважатися пункти 1, 3, адже швидкодія це завжди добре, а не прив'язаність ні до чого, дасть змогу розробити бібліотеку, як окремий модуль, що буде просто розповсюджувати, підтримувати та використовувати.

Основними мовами, що розглядалися були:
\begin{itemize}
	\item C та C++
	\item Python
	\item Common Lisp
\end{itemize}

Мови \textbf{С} та \textbf{С++} загалом є не тільки мовами системного програмування. Їх з успіхом використовували для розробки різноманітних додатків. Основними перевагами цих мов є швидкодія отриманої програми. Недоліком є складність та витратність розробки програмного забезпечення, що працює зі складними структурами даних. Також не дуже зручно програмувати багатопоточність у програмах, якщо використовувати такі мови.

\textbf{Python} --- досить широко використовувана, скриптова, інтерпретована, високорівнева мова програмування. Вона доступна для більшості платформ. Вона підтримує концеп функціонального програмування, зокрема у ній функції є об'єктами першого класу (тобто можуть передаватися як змінна та будти збереженими у змінну). Також Python має багато вбудованих типів, що гарно підходять для данної задачі. Серед мінусів слід зазначити, що ця мова інтерпретована. Існування інтерпретатора накладає свої мінуси, серед яких зменшення швидкодії та залежність від додаткових програмних засобів.

\textbf{Common Lisp} --- це діалект Lisp-у, що набув значного розповсюдження у програмному світі. Існує реалізації під більшість платформ. Мова високорівнева, мультипарадигменна, компільована. Основні переваги для роботи з деревами ця мова має через те, що вона є Lisp-мовою, тобто вона створена для маніпулювання списками структура даних. У програмування дерева найчастіше подаються у вигляді списку списків, тому більшість алгоритмів для работи з деревами гарно програмуються на Lisp-і. Серед мінусів слід зазнасити невелику популярність (якщо порівнювати з іншими не-Lisp-ами), та невелику кількість бібліотек.

Загалом, серед розглянутого, Common Lisp був би найкращим вибором, якщо якость подолати його мінуси.
\subsubsection{JVM}
Віртуальна машина Java --- набір комп'ютерних програм та структур даних, що використовують модель віртуальної машини для виконання інших комп'ютерних програм чи скриптів. JVM використовує байт-код Java, який як правило, але не завжди генерується з вихідних кодів мови програмування Java; віртуальну машину також застосовують для виконання коду, згенерованого з інших мов програмування. JVM доступна для всіх основних сучасних платформ, тому про програми, що скомпільовані у Java байткод теоретично можна сказати ``Написано один раз, працює скрізь''.

Основними мовами для розглядання були:
\begin{itemize}
	\item Groovy
	\item Scala
	\item Clojure
\end{itemize}

\textbf{Groovy}
Groovy — об'єктно-орієнтована динамічна мова програмування, що працює в середовищі JRE. Мова Groovy запозичла деякі корисні якості Ruby, Haskell і Python, але створена для роботи всередині віртуальної машини Java (JVM) і підтримує тісну інтеграцію з Java програмами.

Оскільки Groovy працює в середовищі JRE, то саме Java є основним так би мовити конкурентом. Розробники недвозначно акцентують увагу в різноманітних описах на тому, що дана мова дуже схожа на Java і використовує її інфраструктуру, відповідно потребує мінімум зусиль для вивчення.

\textbf{Scala} --- мультипарадигмова мова програмування, що поєднує властивості об'єктно-орієнтованого та функційного програмування.

Програми на Scala запускаються на віртуальній машині Java вище версії 1.5 за умови включення до дистрибутиву scala-library.jar. Scala сумісна із існуючими програмами на Java, тобто код Scala може викликатися із Java-програм і навпаки. Існує реалізація для платформи .NET, але вона підтримується значно менше можливостей. Дистрибутив Scala, включаючи компілятор і бібліотеки, випущено під BSD похідною ліцензією.

\textbf{Clojure} --— сучасний діалект мови програмування Lisp. Це мова загального призначення, що підтримує інтерактивну розробку, зорієнтовану на функціональне програмування, спрощує багатотредове програмування, та містить риси сучасних скриптових мов.

Clojure працює на Java Virtual Machine і Common Language Runtime. Як і інші Lisp-подібні мови, Clojure розглядає код як дані і має потужну систему макросів. Clojure, бібліотеки и runtime-компоненти розповсюджується в рамках ліцензії Eclipse Public License.

Clojure був розроблений з думкою про сучасний Lisp для функціонального програмування, розрахований на інтеграцію з розповсюдженою платформою Java й розроблений для паралельного програмування.

Підхід Clojure до паралельності характеризується концепцією тотожностей, що представляють серію незмінних станів протягом часу. Оскільки стани є незмінними значеннями, будь-яка кількість обробників може паралельно обробляти їх, і конкуренція зводиться до питання керування змінами від одного стану до іншого. З цією метою, Clojure надає декілька типів змінюваних посилань, кожен з яких має добре визначену семантику переходу між станами.

Переваги Clojure для розроблювального програмного забезпечення полягають у тому, що це Lisp, що ця мова має нахил до паралельного програмування, та працює на JVM. Тобто бібліотека на такій мові зможе використовувати усі можливості Java, а також інші програми, написані на Java, зможуть використовувати цю бібліотеку.
\subsubsection{.NET}
Microsoft .NET --- платформа від фірми Microsoft для створення як звичайних програм, так і веб-застосунків. Багато в чому є продовженням ідей та принципів, покладених в технологію Java. Одною з ідей .NET є сумісність служб, написаних різними мовами. Хоча ця можливість рекламується Microsoft як перевага .NET, платформа Java має таку саму можливість.

.NET — крос-платформова технологія, існує реалізація для платформи Microsoft Windows, FreeBSD (від Microsoft) і варіант технології для ОС Linux в проекті Mono (в рамках угоди між Microsoft з Novell), DotGNU. .NET поділяється на дві основні частини — середовище виконання (по суті віртуальна машина) та інструментарій розробки.

Основні мови, що розглядалися:
\begin{itemize}
	\item C\#
	\item C++/CLI
	\item F\#
\end{itemize}

\textbf{C\#} --— об'єктно-орієнтована мова програмування з безпечною системою типізації для платформи .NET. Розроблена Андерсом Гейлсбергом, Скотом Вілтамутом та Пітером Гольде під егідою Microsoft Research.

Синтаксис C\# близький до С++ і Java. Мова має строгу статичну типізацію, підтримує поліморфізм, перевантаження операторів, вказівники на функції-члени класів, атрибути, події, властивості, винятки, коментарі у форматі XML. Перейнявши багато що від своїх попередників — мов С++, Delphi, Модула і Smalltalk — С\#, спираючись на практику їхнього використання, виключає деякі моделі, що зарекомендували себе як проблематичні при розробці програмних систем, наприклад множинне спадкування класів (на відміну від C++).

\textbf{C++/CLI} --— прив'язка мови програмування С++ до середовища програмування .NET фірми Microsoft. Вона інтегрує С++ стандарту ISO з Об'єднаною системою типів (Unified Type System, UTS), що розглядається як частина Загальної мовної інфраструктури (Common Language Infrastructure, CLI). Вона підтримує і сирцевий рівень, і функціональну сумісність виконуваних файлів, скомпільованих із рідного і керованого C++. C++/CLI являє собою еволюцію С++. C++/CLI стандартизований в ECMA як ECMA-372.

Загалом, можна сприймати як просто мову C++, що була розглянута все раніше. Більшість плюсів і мінусів у них співпадають.

\textbf{F\#} --— багатопарадигмова мова програмування, розроблена в підрозділі Microsoft Research і призначена для виконання на платформі Microsoft.NET. Вона поєднує в собі виразність функціональних мов, таких як OCaml і Haskell, з можливостями і об'єктною моделлю .NET. Функційна мова максимально адаптована до використання в .NET Framework, відповідно, вона не заперечує і імперативного підходу.

Протягом тривалого часу F\# існував як дослідницький проект, основним завданням якого було збагатити імперативну мову C\# можливостями, традиційно доступними лише функціональним мовам. Безліччю нововведень C\# 3.0 з VS 2008 завдячує саме йому. Сам по собі F\# не створений з чистого аркуша в Microsoft, в його основу покладено досить популярний OCaml, який, у свою чергу, походить від одної з перших типізованих функціональних мов ML. Попри те, що синтаксично F\# і OCaml досить близькі, вони не еквівалентні: грубо кажучи, перший становить собою підмножину другого, доповнену доступом до властивостей .NET Framework. Однак деякі програми на OCaml можуть бути практично без модифікацій скомпільовані F\#, зворотна компіляція також справедлива, зрозуміло, за відсутності звернень до класів .NET Framework.
\subsubsection{Висновки}
Розглядаючи і порівнюючи доступні платформи у попередньому розділі, можна прийти до деяких висновків, враховуючи їх особливості у контексті поставленої задачі.

Написання бібліотеки без платформи хоч і має свої плюси, однак мінуси у вигляді втрати доступу до великої кількості розробників, що все використовують якусь платформу, важать більше, ніж плюси. До того ж більшість розглянутих мова не дуже підходить до задачі. Вийняток складає мова Common Lisp, що гарно підходить для програмування вибраних методів, однак її популярність ще менша.

Розглядаючи .NET, ми відмітили, що платформа сама по собі досить актуально, але набір мов, що доступні для розробки на цій платформі, також не дуже підходить для розробки системи пошуку у структурах даних гри го.

У свою чергу Java демонструє багато позитивних сторін, що знадобляться для реалізації методів. Вона популярна, багатоплатформена, має значний набір мов програмування. Більшість з них досить своєрідні, однак Clojure --- гарний вибір для розробки бібліотеку.

По-перше ця мова Lisp-подібна, а про переваги для данного випадку таким мов, все було описано. А по-друге вона має значну орієнтацію на паралельне програмування.

Саме тому з усіх розглянутих платформ, найкаще для данної задачі підходить Java. Хоча все було розглянуто основні характеристики деяких мов для Java, розглянемо їх докладніше у наступному розділі, для кращого розуміння їх властивостей.
\subsection{Вибір мови програмування}
\subsubsection{Grooby}
\subsubsection{Scala}
\subsubsection{Clojure}
\subsubsection{Висновки}
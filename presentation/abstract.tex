\annotation{Анотація}
Дипломна робота присвячена розробці математичних та програмних засобів для моделювання процесів конвекції-дифузії з переважанням дифузії.

У рамках роботи проведено аналіз підходів до математичного моделювання процесів конвекції-дифузії, що виникають в прикладних задачах фізики твердого тіла, гідро- та аеродинаміки, біологічних та медичних дослідженнях. Обрано математичну модель задачі з урахуванням переважання дифузних процесів.

Розглянуто способи чисельного моделювання процесів конвекції-дифузії з переважанням дифузії на основі обраної математичної моделі та обгрунтовано використання методу граничних елементів.

Розроблено програмне забезпечення, що реалізовує обрані методи моделювання дифузних процесів.

Ключові слова: процес конвекції-дифузії, дифузія, метод граничних елементів, метод колокації.

\annotation{Abstract}
In this thesis we consider the development of mathematical and software tools for simulation of convection-diffusion processes with diffusion prevalence.

In this paper we perform the analysis of theorethical approaches to convection-diffusion simulation that can be applied to the different problems in solid state physics, hydrodynamics, aerodynamics, biology and medical studies. We choose the mathematical model of this problem subject to prevalence of diffusion.

We discuss the methods of numerical modeling of convection-diffusion processes with diffusion prevalence based on choosed mathematical model and substantiate using of boundary elements method.

We describe software tools for simulation of diffusion processes developed as part of this thesis.

Keywords: convection-diffusion process, diffusion, boundary elements method, collocation method.

\annotation{Аннотация}
Дипломная работа посвящена разработке математических и программных средств для моделирования процессов конвекции-диффузии с преобладанием диффузии.

В рамках работы проведено анализ подходов к математическому моделированию процессов конвекции-диффузии, которые возникают в прикладных задачах физики твердого тела, гидро- и аеродинамики, биологических и медициских исследованиях. Выбрано матматическую модель задачи с учетом преобладания диффузных процессов.

Рассмотрено способы численного моделирования процессов конвекции-диффузии с преобладанием диффузии на основании выбраной математической модели и обосновано использование метода граничных элементов.

Разработано програмное обеспечение, реализующее выбраный метод моделирования диффузных процессов.

Ключевые слова: процесс конвекции-диффузии, диффузия, метод граничных элементов, метод коллокации.

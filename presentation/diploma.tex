\documentclass{diploma}


% makes pdf searchable & copyable
% \usepackage{cmap}

% input encoding
% \usepackage[utf8]{inputenc}

% font encoding
% \usepackage[T2A]{fontenc}

% add languages support
% \usepackage[ukrainian,english]{babel}

\begin{document}
% \maketitlepage{Борисенко Павло Борисович}{КМ-02}{ст. викл. Любашенко Н.Д.}{доцент, к. ф.-м. н. Шубенкова І.А.}{Система паралельного пошуку}

% \assigment{
    % StudentName={Борисенку Павлу Борисовичу},
    % ThesisName={\invcommas{Моделювання процесів конвекції-дифузії з переважанням дифузії}},
    % AdvisorName={ст. викладач Любашенко Наталія Дмитрівна},
    % Order={\invcommas{28}~травня~2014~р.~\No~995-C},
    % ApplicationDate={\invcommas{15}~червня~2014~р.},
    % InputData={\begin{itemize}
            % \item алгоритм методу граничних елементів;
            % \item дані задачі дифузії.
        % \end{itemize}},
    % Contents={\begin{itemize}
            % \item аналіз існуючих методів вирішення задачі;
            % \item вибір омтимального методу;
            % \item програмна реалізація;
            % \item порівняння результатів.
        % \end{itemize}},
    % Graphics={\begin{itemize}
            % \item блок-схеми алгоритмів;
            % \item знімки екранних форм.
        % \end{itemize}},
    % AssigmentDate={\invcommas{01}~жовтня~2013~р.},
    % Calendar={& & & \\},
    % StudentPIB={Борисенко П.Б.},
    % AdvisorPIB={Любашенко Н.Д.}
% }

% set one and half spacing size
% \usepackage{setspace}
% \onehalfspacing
% \renewcommand{\baselinestretch}{1.4}

% indent first paragraph
% \usepackage{indentfirst}

% for math symbols
% \usepackage{amsmath}
% \usepackage{amssymb}

% so tables will stay in place
% \usepackage{float}
% \restylefloat{table}

% include pictures
% \usepackage[pdftex]{graphicx}

% change the settings of counters
% \usepackage{chngcntr}

% change sections formating
% \usepackage{titlesec}
% \titleformat*{\section}{\normalfont \bfseries \center}
% \titleformat*{\subsection}{\normalfont \bfseries}
% \titleformat*{\subsubsection}{\normalfont \bfseries \itshape}
% add dot to the end of numbering
% \titlelabel{\thetitle. }

% write pseudocode examples
% \usepackage{algpseudocode}

% \usepackage{etoolbox}
% \newfontfamily{\codefont}{Menlo}
% \AtBeginEnvironment{algorithmic}{\codefont}

% add option to all lists
% \usepackage{enumitem}
% \setitemize{parsep=0pt}
% \renewcommand{\labelitemi}{--}

% specify page spacing
% \usepackage[top=2cm, bottom=2cm, left=3cm, right=2cm]{geometry}
% some more margins and indents
% \setlength{\evensidemargin}{0.5cm}
% \setlength{\oddsidemargin}{0.5cm}
% \setlength{\textheight}{25cm}
% \setlength{\textwidth}{16cm}
% \setlength{\parindent}{1cm}

% specify font
% \renewcommand{\familydefault}{\rmdefault}

% reformat table and figure numbering
% \renewcommand{\thetable}{\thesection .\arabic{table}}
% \renewcommand{\thefigure}{\thesection .\arabic{figure}}

% don't display subsubsection in TOC
% \setcounter{tocdepth}{2}
% \pagestyle{plain}
% \pagenumbering{arabic}

	\selectlanguage{ukrainian}

    \maketitlepage{Крамаренко Олексій Андрійович}{КП-01}
        {ст. викл. Любашенко Н.Д.}{доцент, к. ф.-м. н. Шубенкова І.А.}
        {Система паралельного пошуку у структурах даних гри Го}
        {асистент Онай М.В. }

    \assigment{
        StudentName={Борисенку Павлу Борисовичу},
        ThesisName={\invcommas{Моделювання процесів конвекції-дифузії з переважанням дифузії}},
        AdvisorName={ст. викладач Любашенко Наталія Дмитрівна},
        Order={\invcommas{28}~травня~2014~р.~\No~995-C},
        ApplicationDate={\invcommas{15}~червня~2014~р.},
        InputData={\begin{itemize}
                \item алгоритм методу граничних елементів;
                \item дані задачі дифузії.
            \end{itemize}},
        Contents={\begin{itemize}
                \item аналіз існуючих методів вирішення задачі;
                \item вибір омтимального методу;
                \item програмна реалізація;
                \item порівняння результатів.
            \end{itemize}},
        Graphics={\begin{itemize}
                \item блок-схеми алгоритмів;
                \item знімки екранних форм.
            \end{itemize}},
        AssigmentDate={\invcommas{01}~жовтня~2013~р.},
        Calendar={& & & \\},
        StudentPIB={Борисенко П.Б.},
        AdvisorPIB={Любашенко Н.Д.}
    }

    \annotation{Анотація}
Дипломна робота присвячена розробці математичних та програмних засобів для моделювання процесів конвекції-дифузії з переважанням дифузії.

У рамках роботи проведено аналіз підходів до математичного моделювання процесів конвекції-дифузії, що виникають в прикладних задачах фізики твердого тіла, гідро- та аеродинаміки, біологічних та медичних дослідженнях. Обрано математичну модель задачі з урахуванням переважання дифузних процесів.

Розглянуто способи чисельного моделювання процесів конвекції-дифузії з переважанням дифузії на основі обраної математичної моделі та обгрунтовано використання методу граничних елементів.

Розроблено програмне забезпечення, що реалізовує обрані методи моделювання дифузних процесів.

Ключові слова: процес конвекції-дифузії, дифузія, метод граничних елементів, метод колокації.

\annotation{Abstract}
In this thesis we consider the development of mathematical and software tools for simulation of convection-diffusion processes with diffusion prevalence.

In this paper we perform the analysis of theorethical approaches to convection-diffusion simulation that can be applied to the different problems in solid state physics, hydrodynamics, aerodynamics, biology and medical studies. We choose the mathematical model of this problem subject to prevalence of diffusion.

We discuss the methods of numerical modeling of convection-diffusion processes with diffusion prevalence based on choosed mathematical model and substantiate using of boundary elements method.

We describe software tools for simulation of diffusion processes developed as part of this thesis.

Keywords: convection-diffusion process, diffusion, boundary elements method, collocation method.

\annotation{Аннотация}
Дипломная работа посвящена разработке математических и программных средств для моделирования процессов конвекции-диффузии с преобладанием диффузии.

В рамках работы проведено анализ подходов к математическому моделированию процессов конвекции-диффузии, которые возникают в прикладных задачах физики твердого тела, гидро- и аеродинамики, биологических и медициских исследованиях. Выбрано матматическую модель задачи с учетом преобладания диффузных процессов.

Рассмотрено способы численного моделирования процессов конвекции-диффузии с преобладанием диффузии на основании выбраной математической модели и обосновано использование метода граничных элементов.

Разработано програмное обеспечение, реализующее выбраный метод моделирования диффузных процессов.

Ключевые слова: процесс конвекции-диффузии, диффузия, метод граничных элементов, метод коллокации.


    \tableofcontents


	\intro
Го -- стародавня китайська стратегічна гра на двох гравців. Вона є антагоністичною грою з повною інформацією. В Го грають два гравці -- ``Чорні'' та ``Білі''. Вони по черзі розміщують на дошці, що складається з перетину 19 на 19 ліній, камені свого кольору. Го територіальна гра, тобто гравець, що під кінець гри має більшу територію -- виграє. Як і для багатьох інших подібних ігор, були спроби створити комп'ютерні програми, що гарно грають в Го, але це виявилося справжнім викликом для програмістів. Складність обчислення партій в Го на кілька порядків більша за шахи. На кожному кроці можливі близько 200—300 ходів, статична ж оцінка життя груп каменів фактично неможлива. Одним ходом тут можна цілком зіпсувати всю гру, навіть коли решта ходів були дуже добрі. Тому програми для гри в Го не використовують таких алгоритмів, як шахові програми, а замість цього зазвичай мають кілька десятків модулів для оцінки різних аспектів гри і під час аналізу намагаються використовувати ті ж самі поняттями, що й люди. Попри це вони і далі грають дуже слабко та програють навіть не дуже сильним аматорам.

% Перша програма для гри в Го була написана Альбертом Зобріст в 1968 році як частина дисертації по розпізнаванню образів. Він використав функцію впливу для оцінкі території і Зобріст-хешування для виявлення ситуацій Ко.

% Протягом довгого часу було широко поширена думка, що комп'ютерне Го являє собою проблему, що в корені відрізняється від комп'ютерних шахів, оскільки вважалося, що методи, що спираються на швидкий глобальний пошук дадуть гірший результат, порівняно з експертою оцінкою гри. Саме тому більша частина зусиль в області розвитку комп'ютерних програм того часу була зосереджена на експертних системах. Були спроби об'єднати такі системи з локальним покушом, аби дати можливість програмі відповідати на питання тактичного характеру. Результатом цього були створені програми, які мали здатність оброблювати багато різних локальних ситуацій, але які мали дуже виражені недоліки у загальній тактиці гри. Крім того, ці класичні програми не отримали майже нічого від збільшення доступних обчислювальних потужностей і прогрес в цій області в цілому йшов дуже повільно.

% Кілька дослідників зрозуміли потенціал імовірнісних методів і передбачили, що вони будуть домінувати у комп'ютерному Го. Але у той же час вони розуміли те, що гарну ігрову програму можна буде розробити тільки в далекому майбутньому, в результаті фундаментальних досягнень в загальній технології штучного інтелекту. Навіть написати програму, здатну автоматично визначати переможця готової грі було не тривіальним завданням.

% Поява програм, заснованих на пошуку Монте-Карло, кардинально змінила ситуацію, хоча розрив між професійними гравцями і найсильнішими програми все ще залишається значним. Останні розробки у цих методах пошуку по деревах і машинному навчанню принесли кращі результати для програм, що грають в Го на маленькій дошці 9x9. У 2009 році з'явилися перші подібні програми, що могли змагатися з низькими данами на дошках розміром 19х19.

Усі програми, що грають в Го повинні оперувати деякими структурами даних (наприклад такими, що репрезентують поточний стан дошки або гри). Більшість з таких програм засновані на створенні великої кількості різних партій, та подальшому їх аналізі. Саме тому питання пошуку у подібних структурах даних дуже актуальне, адже його оптимізація корінним чином вплине на роботу та якість програм, що грають в Го.


	\newpage
\section{Аналіз існуючих рішень}
\subsection{Огляд існуючих алгоритмів}
Єдиний вибір, що повинна зробити програма, що грає в го, це куди поставити наступний камінь. Однак, цей вибір ускладнюється тим, що навіть один камінь може дуже сильно впливати на ситуацію на дошці в цілому. У той же час не можно забувати про об'єднання каменів --- їх групи, та про взаємодію цих груп між собою. Для вирішення цієї проблеми використовуються різні підходи. Розглянемо деякі з них.
\subsubsection{Мінімаксний пошук по дереву варіантів}
Мінімаксний пошук може використовуватися для того, щоб моделювати велику кількість різних партій. Загалом, алгоритм простий: спочатку алгоритм по черзі грає усі варіанти ходів до деякого моменту. Потім використовується функція оцінки позицій, для вираховування того, наскільки поточна позиція гарна для кожного гравця. Далі, на основі цього зваженого дерева ходів, робиться розрахунок оптимальної стратегії для одного з гравцій --- вибираються ті ходи, що дають найбільше переваги цьому гравцю.

Хоча цей метод був досить ефективним для шах, він не дуже підходить для го. По-перше, через те, що досі не було створено відповідної функції оцінки для позиції в го. Гра го все ще не формалізована математично, тому поточні функції оцінки партії запрограмовані робити висновки як люди. Тобто програмісти намагалися навчити свої програми грати, як вони самі. По-друге, через те, що для партії в го притаманний великий фактор розгалуження. У кожний момент гри для кожного з гравців існує дуже багато коректних ходів. Також самі партії в го довші, ніж у шахах. Саме тому такі методи дуже обчислювально коштовні. На данний момент програми, що використовують подібні алгоритми можуть грати тількі на дошках менших ніж 9x9.

Є декілька технік, що дозволяють значно спростити обчислення для цього методи. Такі методи базуються на відсіченні піддерев за деякими правилами та дозволяють значно зменшити фактор розгалуження не послаблюючи алгоритм. Також для оптимізації роботи з деревом у подібних алгоритмах доцільно використовувати хешування стану дошки. Найкращим методом хешування для го є метод Зобріст-хешування. Він базується на використанні функції XOR до поточного стану усіх клітинок на дошці. Цей метод гарно себе показав, по відношенню до го, адже він дозволяє легко рахувати хеш нового положення дошки, маючи попередній хеш та послідовність ходів, тобто дозволяє не перераховувати його з самого початку. Також існує підхід по відсіченню піддерев, використовуючи деякі припущення про саму партію. Наприклад, надавати пріорітет ходам, що намагаються врятувати групу каменів, або знижувати пріорітет на ділянках дошки, що все є досить сильними для одного з гравців. Але такі варіанти створюють небезпеку не враховування деяких вкрай важливих ходів, які б кардинально змінили течію гри, тому такі підходи є досить ризиковими.

Результати ігор між різними комп'ютерними програми, що використовують різні методи, дозволило прийти до висновку, що кращий спосіб гри в го --- поєднання методів порівняння зі зразком із методами швидкого локалізованого тактичного пошуку. Розглянемо їх у наступних розділах.
\subsubsection{Порівняння зі зразком}
Методи, що використовують порівняння зі зразком маніпулюють послідовністю ходів, що є прийнятною для обох гравців. Це відомі маленькі шматочки з яких найчастіше складаються локальні ситуації. Такі послідовності добре вивчені і обґрунтовані, тому достатньо їх правильно використовувати всередині гри. Пошук цих зразків є дуже важливим як для гравців-людей, так я для програм, що грають в го. Розглянемо один із можливих алгоритмів пошуку таких зразків в іграх го.

Виберемо деяку зону на дошці, яку будемо вважати зразком. Наприклад квадрат 5x5. Потім потрібно його захешувати, тобто представити у вигляді int64 числа. Однак перец цим потрібно привести його до деякого базового вигляду. Адже навіть однакові зразки з точністю до повороту або симетрії будуть виглядати різними на дошці, бо не будуть співпадати поклітинно. Вього вісім ``різних'' позицій будут однаковими. Чотири повороти(на 0, 90, 180, 270 градусів відповідно) і 4 повороти віддзеркаленого зразка. Нехай базовий вигляд буде вигляд, у якого хеш найменше число. Тоді порахувавши 8 хешів і обравши менший, ми зведемо всі подібні зразки до одного. Далі, якщо зберігти багато подібних зразків у базу, єю можна будет користуватися, шукаючи у ній частинки поточної гри. З великою ймовірністю, декілька початкових каменів дадуть змогу знайти відповідний гарний шаблон, який і треба будет далі відіграти програмі.

Говорячи про ймовірність, можна згадати ще один клас методів гри у го --- методи, засновані на вірогідності. Він використовує напрацювання попереднього методу, додаючи до них цікавий алгоритм навчання.
\subsubsection{Методи, засновані на вірогідності}
Базуючись на попередньому методі, можна отримати базу зразків ігор досвідчених гравців. Використовуючи цю базу, можна отримати розподіл ймовірностей ходів для професійних ігор, який можна використовувати для відтворення цих ходів у окремій програмі. Цей розподіл можна використовувати не тільки для програми, що грає в го, але й в якості навчального посібника для гравців у го. Цей метод має дві основні складові: 1) схема вилучення шаблону з експертних партій гри (реалізовано у попередньому пункті) 2) байесовський алгоритм навчання, який навчається розподілу аналізуючи ходи у локальному місці дошки.

Якщо спробувати комп'ютер мислити як людина, щоб аналізувати локальну позицію на дошці, то можна отримати методи, засновані на базі знань.
\subsubsection{Методи, засновані на базі знань}
Якщо використовувати все ті ж самі шаблони, згенеровані методом порівняння зі зразком, але додати до них інтелект програміста, то вийде досить сильна програма для гри в го. Під інтелектом програміста, мається на увазі можливість вирішувати локальну позицію у шаблоні використовуючи деякий набор еврістик. Програмісту достатньо тільки перевести ці правила в комп'ютерний код та використати пошук за зразком, щоб знаходити ситуації, де ці правила доречні. Основний недолік - складність цих правил, а точніше можливість програмування їх, залежить насамперед від здатності та навику гри в го самого програміста. Зважаючи на те, що математичного апарату для подібної роботи нема, кожен програміст намагається навчити свою програму грати, як він. Тому найчастіше програми мають більше сотні модулів, що вираховують найкращий хід кожен окремо для своєї ситуації. Однак такі методи страждають від проблем, аналогічних попереднім --- нерозуміння глобальної ситуації. Це призводить до того, що вони роблять помилки у стратегічному плані. Відомо, що можливо програти гру, якщо у вирішальний момент обрати неправильний хід.

Наступний метод вирішує проблемні питання у стратегії повним ігноруванням її, як і самих правил. Методи Монте-Карло використовуються у багатьох галузях знать, таких як математична статистика і теоретична фізика. Знайшли вони застосування і в алгоритмах для гри у го.
\subsubsection{Методи Монте-Карло}
Одією з головних альтернатив використанню жорстко запрограмованих методів пошуку --- використовувати методи Монте-Карло. Якщо говорити про го, то цей метод полягає в наступному:  згенеруємо список потенційних ходів, які ми хочемо перевірити; для кожного такого ходу зіграємо тисячу випадкових партій (тобто наступних випадкових ходів до остаточного кінця гри --- стану на дошці, при якому можливо точно визначити переможця); виберемо з списку ходів той, при якому найбільша кількість з випадкових партій виграна програмою.

Перевага цього методу полягає в тому, що він потребує дуже малих знань про предметну область, у якій працює, недоліком є те, що він використовує більше пам'яті та процесорного часу. Однак через те, що ходи генеруються випадковим чином, ми можемо неправильно оцінити якість ходу. Наприклад, якщо у відповідь на якийсь хід буде згенеровано 100 ходів-відповідей, у більшості з який перший гравець виграє, ми будемо оцінювати цей хід як дуже сильний. Однак існує можливість того, що є дуже добра відповідь на наш хід, яка зведе нашу перевагу на нівець, однак, через те, що ми не згенерували цей хід, ми про це не дізнаємося. В результаті цього, програма буде сильна в загальному стратегічному сенсі, однак буде дуже слаба тактично. Цю проблему можна вирішити, якщо додати деякі проблемно-орієнтовні знання в генерацію ходів та підвищити глибину пошуку.

У 2006 році був створений новий пошуковий алгоритм, що був використаний для гри на дошках розміру 9x9 та зарекомендував себе дуже гарно. Він називається Upper Confidence Bounds algorithm і базується на методі Монте-Карло. Алгоритм UCT змінює правила за якими визначається важливість ходів у дереві пошуку. Він вибирає те піддерево, у якому вірогідність перемоги більше 50\%. Якщо ж не існує такого піддерева, то вибір робиться навмання.
\subsection{Огляд існуючих програмних продуктів}
Кожна програма, що грає в го повинна вміти ефективно шукати ходи/партії у своїй внутрішній базі данних. Однак партії з го мають важливе навчальне значення самі по собі. Тому існують програми, що єдиною своєю метою ставлять роботу з партіями го. Вони об'єднують багато партій в одну базу даних і маніпулюють єю. Розглянемо деякі з них.
\subsubsection{Kombilo}
Kombilo --- програма, що працює з базою го ігор. Основне завдання такої програми --- пошук деяких підпослідовностей в колекції SGF-ігор (наприклад пошук усіх ігор з деяким початком). Також ця програма дозволяє шукати по деяким властивостям ігор (таким, як гравці, події, дати).

Особливості програми:
\begin{itemize}
	\item Можливість пошуку як по повному ігровому поля, так і по позиціями у куті або на стороні дошки. Пошук ведеться також враховуючи симетрію та поворот. Існує можливість інвертувати пошук по кольору гравців, тобто поміняти їх місцями.
	\item Kombilo також комплектується повним редактором SGF: тобто існує можливість редагувати файли, коментувати їх та інше. Також редактор дозволяє бачити дерево варіантів ігри. Він дозволяє повертати/віддзеркалювати ігри.
	\item Kombilo також має у собі механізм по відображенню посилань. Цей механізм додає підказки до ігор, що впізнає. На поточний момент база цих посилань налічує 2000 записів.
	\item Можливо використовувати складні запити, використовуючи напряму SQL-базу, що зберігає розпарсені ігри.
	\item Можливо застосувати будь-яку комбінацію пошукових запитів, та у будь-який момент отримати список поточних ігор.
\end{itemize}

Говорячи про реалізацію програми, можна відзначити що вона написана на Python, тому може вважатися кроссплатформеною. Основне ж ядро, що присвячене пошуку, також написане на С++. Компілюючи його у модуль та підключаючи до основної програми ми отримаємо збільшення швидкодіє більше, ніж у два рази. Також це відкрита програма, і програмний код доступний для читання та аналізу.
\subsubsection{Master Go}
Це комерційна програма, що працює з базими даних гри го і призначена для пошуку поширених початків та розвитків ігор (вивченні фусекі і дзосекі). База даних містить 53059 професійних ігор у власному форматі. Оновлення доступні для всіх зареєстрованих користувачів.  Можливо додавати ігри в базу, що постачається з MasterGo шляхом придбання цих ігор в Японії, Китаї і Кореї і копіюванням їх в базу. Також Master Go працює тільки на операційній системі MS Windows.

Якщо порівнювати цю програму з Kombilo, та можна помітити декілька речей: 1) Master Go, комерційна та платна програма 2) вона швидша, адже написана безпосередньо під одну платформу 3) вона використовує внутрішній формат даних, що також є не дуже зручним.
\subsubsection{BiGo Assistant}
BiGo Assistant --- програма для роботи з базами даних фусекі та джосекі, створена для гравців, що хочуть підняти свій рівень гри в го.

Основні особливості:
\begin{itemize}
	\item Можливість не тільки вивчати ігри го в цілому, але й переглядати ігри професіоналів у декільках режимах
	\item Можливість вивчати початковий розвиток партії в го (фусекі) за прикладом бази даних професійних ігор
	\item Також присутня база даних джосекі
	\item Можливість аналізувати власні початки різних ігри
	\item Динамічний інтерфейс --- можливість створення будь-якої кількості вікон з різними іграми
	\item Можливість роздруковувати партії
	\item Можливість застосувати до дошки всі 8 можливих перетворень (симетрія та поворот)
	\item Програма відслідковує дублікати ігор
	\item Можливість збирати статистику по списку ігор у вигляді статистики по варіаціям можливих рухів або оціночну статистику позицій у грі обох гравців
	\item Пошук по заданій позиції (по одній або декільком частинам дошки)
\end{itemize}

Програма має багато можливостей, однак через те, що цікавим представляється тільки метод пошуку, ця програма має одну важливу відмінність від усліх інших: вона дозволяє шукати по декільком частинам дошки.

Після аналізу можливостей існуючих рішень по темі дипломної роботи, можна ставити задачу по розробці власної програми, враховуючи 
\subsection{Постановка задачі}
Було вибрано реалізувати бібліотеку, яка б містила деякі з запропонованих методів пошуку, а саме:
\begin{itemize}
	\item Мінімаксний пошук по дереву
	\item Порівняння зі зразком
	\item Метод Монте-Карло
\end{itemize}
Ці методи були вибрані тому, що вони самі по собі не залежать від гри, а тількі від внутрішнього представлення цієї гри. Також цим методам, на відміну від інших, не треба ``пояснювати'' правила го, вони можут працювати окремо і бути використані у іншому модулі для аналогічного пошуку у подібних структурах даних інших ігор.

Створити бібліотеку було вибрано тому, що це гарний спосіб інкапсулювати декілька реалізацій методу пошуку у структурах даних гри го. Також у такий спосіб інші програми або модулі можуть використовувати ці реалізації, більш піклуючись про інші аспекти гри.

	% \newpage
\section{Вибір засобів реалізації}
У попередньому розділі було прийнято рішення розроблювати бібліотеку, що буде дозволяти шукати у структурах даних гри го декількома різними способами. Ця бібліотеку може бути використана програмами, що аналізують партії в го, збирають статистику з партій або ж шукають деякі закономірності у партіях. Щоб ця бібліотека була корисною і використовувалася, важливо вибрати правильну платформу та мову програмування для неї.

Основними властивостями розроблювальної бібліотеки повинні бути:
\begin{itemize}
	\item Платформонезалежність --- це зробить бібліотеку більш поширеною та зручною.
	\item Можливість використання у багатьох популярних мовах програмування (загалом, це доповнення до попереднього пункту).
	\item Простота реалізації. Бажано, щоб внутрішні структури даних, були змодельовані використовуючи вбудовані структури даних мови.
	\item Бажано, щоб мова підтримувала функціональний підхід програмування. Функціональний підхід гарно себе зарекомендував при роботі з штучним інтелектом або якимось складним аналізом даних.
\end{itemize}

Розглянемо деякі можливі платформи для поставленої задачі.
\subsection{Вибір платформи}
Вибір платформи для програми, це дуже важливе рішення, адже платформа одразу не тільки поставить обмеження на доступні мови програмування але і додасть свої плюси та мінуси, до розроблюваного програмного забезпечення. Загалом, дуже загально, можна разбити платформи на три пункти:
\begin{itemize}
	\item Без використання платформи
	\item .NET
	\item Java
\end{itemize}

Розглянемо їх по черзі оцінюючи доступні мови програмування для кожної платформи та їх плюси і мінуси.
\subsubsection{Без платформи}
Найочевиднішим вибором буде не використовувати ніяку платформу. Адже навіщо ускладнювати, якщо можна зробити простіше. Однак програмування, використовуючи платформу, таку як Java або .NET, має свої значні переваги, серед яких більшість плюсів надається єко-системою платформи. Однак при програмування бібліотеки методів пошуку, не обов'язково використовувати якусь платформу, тож розглянемо цей варіант.

Основними перевагами написання програми без використання будь-якої платформи є:
\begin{itemize}
	\item Швидкодія (не завжди)
	\item Відсутність прошарків між програмою та ОС (найчастіше)
	\item Не прив'язаність ні до яких інструментів
	\item Можливість написати простий standalone-додаток
\end{itemize}

Для бібліотеки важливими можуть вважатися пункти 1, 3, адже швидкодія це завжди добре, а не прив'язаність ні до чого, дасть змогу розробити бібліотеку, як окремий модуль, що буде просто розповсюджувати, підтримувати та використовувати.

Основними мовами, що розглядалися були:
\begin{itemize}
	\item C та C++
	\item Python
	\item Common Lisp
\end{itemize}

Мови \textbf{С} та \textbf{С++} загалом є не тільки мовами системного програмування. Їх з успіхом використовували для розробки різноманітних додатків. Основними перевагами цих мов є швидкодія отриманої програми. Недоліком є складність та витратність розробки програмного забезпечення, що працює зі складними структурами даних. Також не дуже зручно програмувати багатопоточність у програмах, якщо використовувати такі мови.

\textbf{Python} --- досить широко використовувана, скриптова, інтерпретована, високорівнева мова програмування. Вона доступна для більшості платформ. Вона підтримує концеп функціонального програмування, зокрема у ній функції є об'єктами першого класу (тобто можуть передаватися як змінна та будти збереженими у змінну). Також Python має багато вбудованих типів, що гарно підходять для данної задачі. Серед мінусів слід зазначити, що ця мова інтерпретована. Існування інтерпретатора накладає свої мінуси, серед яких зменшення швидкодії та залежність від додаткових програмних засобів.

\textbf{Common Lisp} --- це діалект Lisp-у, що набув значного розповсюдження у програмному світі. Існує реалізації під більшість платформ. Мова високорівнева, мультипарадигменна, компільована. Основні переваги для роботи з деревами ця мова має через те, що вона є Lisp-мовою, тобто вона створена для маніпулювання списками структура даних. У програмування дерева найчастіше подаються у вигляді списку списків, тому більшість алгоритмів для работи з деревами гарно програмуються на Lisp-і. Серед мінусів слід зазнасити невелику популярність (якщо порівнювати з іншими не-Lisp-ами), та невелику кількість бібліотек.

Загалом, серед розглянутого, Common Lisp був би найкращим вибором, якщо якость подолати його мінуси.
\subsubsection{JVM}
Віртуальна машина Java --- набір комп'ютерних програм та структур даних, що використовують модель віртуальної машини для виконання інших комп'ютерних програм чи скриптів. JVM використовує байт-код Java, який як правило, але не завжди генерується з вихідних кодів мови програмування Java; віртуальну машину також застосовують для виконання коду, згенерованого з інших мов програмування. JVM доступна для всіх основних сучасних платформ, тому про програми, що скомпільовані у Java байткод теоретично можна сказати ``Написано один раз, працює скрізь''.

Основними мовами для розглядання були:
\begin{itemize}
	\item Groovy
	\item Scala
	\item Clojure
\end{itemize}

\textbf{Groovy}
Groovy — об'єктно-орієнтована динамічна мова програмування, що працює в середовищі JRE. Мова Groovy запозичла деякі корисні якості Ruby, Haskell і Python, але створена для роботи всередині віртуальної машини Java (JVM) і підтримує тісну інтеграцію з Java програмами.

Оскільки Groovy працює в середовищі JRE, то саме Java є основним так би мовити конкурентом. Розробники недвозначно акцентують увагу в різноманітних описах на тому, що дана мова дуже схожа на Java і використовує її інфраструктуру, відповідно потребує мінімум зусиль для вивчення.

\textbf{Scala} --- мультипарадигмова мова програмування, що поєднує властивості об'єктно-орієнтованого та функційного програмування.

Програми на Scala запускаються на віртуальній машині Java вище версії 1.5 за умови включення до дистрибутиву scala-library.jar. Scala сумісна із існуючими програмами на Java, тобто код Scala може викликатися із Java-програм і навпаки. Існує реалізація для платформи .NET, але вона підтримується значно менше можливостей. Дистрибутив Scala, включаючи компілятор і бібліотеки, випущено під BSD похідною ліцензією.

\textbf{Clojure} --— сучасний діалект мови програмування Lisp. Це мова загального призначення, що підтримує інтерактивну розробку, зорієнтовану на функціональне програмування, спрощує багатотредове програмування, та містить риси сучасних скриптових мов.

Clojure працює на Java Virtual Machine і Common Language Runtime. Як і інші Lisp-подібні мови, Clojure розглядає код як дані і має потужну систему макросів. Clojure, бібліотеки и runtime-компоненти розповсюджується в рамках ліцензії Eclipse Public License.

Clojure був розроблений з думкою про сучасний Lisp для функціонального програмування, розрахований на інтеграцію з розповсюдженою платформою Java й розроблений для паралельного програмування.

Підхід Clojure до паралельності характеризується концепцією тотожностей, що представляють серію незмінних станів протягом часу. Оскільки стани є незмінними значеннями, будь-яка кількість обробників може паралельно обробляти їх, і конкуренція зводиться до питання керування змінами від одного стану до іншого. З цією метою, Clojure надає декілька типів змінюваних посилань, кожен з яких має добре визначену семантику переходу між станами.

Переваги Clojure для розроблювального програмного забезпечення полягають у тому, що це Lisp, що ця мова має нахил до паралельного програмування, та працює на JVM. Тобто бібліотека на такій мові зможе використовувати усі можливості Java, а також інші програми, написані на Java, зможуть використовувати цю бібліотеку.
\subsubsection{.NET}
Microsoft .NET --- платформа від фірми Microsoft для створення як звичайних програм, так і веб-застосунків. Багато в чому є продовженням ідей та принципів, покладених в технологію Java. Одною з ідей .NET є сумісність служб, написаних різними мовами. Хоча ця можливість рекламується Microsoft як перевага .NET, платформа Java має таку саму можливість.

.NET — крос-платформова технологія, існує реалізація для платформи Microsoft Windows, FreeBSD (від Microsoft) і варіант технології для ОС Linux в проекті Mono (в рамках угоди між Microsoft з Novell), DotGNU. .NET поділяється на дві основні частини — середовище виконання (по суті віртуальна машина) та інструментарій розробки.

Основні мови, що розглядалися:
\begin{itemize}
	\item C\#
	\item C++/CLI
	\item F\#
\end{itemize}

\textbf{C\#} --— об'єктно-орієнтована мова програмування з безпечною системою типізації для платформи .NET. Розроблена Андерсом Гейлсбергом, Скотом Вілтамутом та Пітером Гольде під егідою Microsoft Research.

Синтаксис C\# близький до С++ і Java. Мова має строгу статичну типізацію, підтримує поліморфізм, перевантаження операторів, вказівники на функції-члени класів, атрибути, події, властивості, винятки, коментарі у форматі XML. Перейнявши багато що від своїх попередників — мов С++, Delphi, Модула і Smalltalk — С\#, спираючись на практику їхнього використання, виключає деякі моделі, що зарекомендували себе як проблематичні при розробці програмних систем, наприклад множинне спадкування класів (на відміну від C++).

\textbf{C++/CLI} --— прив'язка мови програмування С++ до середовища програмування .NET фірми Microsoft. Вона інтегрує С++ стандарту ISO з Об'єднаною системою типів (Unified Type System, UTS), що розглядається як частина Загальної мовної інфраструктури (Common Language Infrastructure, CLI). Вона підтримує і сирцевий рівень, і функціональну сумісність виконуваних файлів, скомпільованих із рідного і керованого C++. C++/CLI являє собою еволюцію С++. C++/CLI стандартизований в ECMA як ECMA-372.

Загалом, можна сприймати як просто мову C++, що була розглянута все раніше. Більшість плюсів і мінусів у них співпадають.

\textbf{F\#} --— багатопарадигмова мова програмування, розроблена в підрозділі Microsoft Research і призначена для виконання на платформі Microsoft.NET. Вона поєднує в собі виразність функціональних мов, таких як OCaml і Haskell, з можливостями і об'єктною моделлю .NET. Функційна мова максимально адаптована до використання в .NET Framework, відповідно, вона не заперечує і імперативного підходу.

Протягом тривалого часу F\# існував як дослідницький проект, основним завданням якого було збагатити імперативну мову C\# можливостями, традиційно доступними лише функціональним мовам. Безліччю нововведень C\# 3.0 з VS 2008 завдячує саме йому. Сам по собі F\# не створений з чистого аркуша в Microsoft, в його основу покладено досить популярний OCaml, який, у свою чергу, походить від одної з перших типізованих функціональних мов ML. Попри те, що синтаксично F\# і OCaml досить близькі, вони не еквівалентні: грубо кажучи, перший становить собою підмножину другого, доповнену доступом до властивостей .NET Framework. Однак деякі програми на OCaml можуть бути практично без модифікацій скомпільовані F\#, зворотна компіляція також справедлива, зрозуміло, за відсутності звернень до класів .NET Framework.
\subsubsection{Висновки}
Розглядаючи і порівнюючи доступні платформи у попередньому розділі, можна прийти до деяких висновків, враховуючи їх особливості у контексті поставленої задачі.

Написання бібліотеки без платформи хоч і має свої плюси, однак мінуси у вигляді втрати доступу до великої кількості розробників, що все використовують якусь платформу, важать більше, ніж плюси. До того ж більшість розглянутих мова не дуже підходить до задачі. Вийняток складає мова Common Lisp, що гарно підходить для програмування вибраних методів, однак її популярність ще менша.

Розглядаючи .NET, ми відмітили, що платформа сама по собі досить актуально, але набір мов, що доступні для розробки на цій платформі, також не дуже підходить для розробки системи пошуку у структурах даних гри го.

У свою чергу Java демонструє багато позитивних сторін, що знадобляться для реалізації методів. Вона популярна, багатоплатформена, має значний набір мов програмування. Більшість з них досить своєрідні, однак Clojure --- гарний вибір для розробки бібліотеку.

По-перше ця мова Lisp-подібна, а про переваги для данного випадку таким мов, все було описано. А по-друге вона має значну орієнтацію на паралельне програмування.

Саме тому з усіх розглянутих платформ, найкаще для данної задачі підходить Java. Хоча все було розглянуто основні характеристики деяких мов для Java, розглянемо їх докладніше у наступному розділі, для кращого розуміння їх властивостей.
\subsection{Вибір мови програмування}
\subsubsection{Grooby}
\subsubsection{Scala}
\subsubsection{Clojure}
\subsubsection{Висновки}

	% \newpage
\section{Опис розроблених програмних засобів}
\subsection{Алгоритм 1}
\subsection{Алгоритм 2}

	% \newpage
\section{Аналіз розроблених програмних засобів}
\subsection{Аналіз швидкодії алгоритмів та якості пошуку}
\subsection{Рекомендації щодо подальшого вдосконалення}

	% \newpage
\section{OХОРОНА ПРАЦІ}
Дана дипломна робота передбачає розробку програмного засобу для пошуку у структурах даних гри Го.  Розробка даної програми відбувається в кімнаті офісу на чотирьох осіб, у кожної з яких є робоче місце на один комп’ютер.

Правильно організована робота по забезпеченню безпеки праці підвищує дисциплінованість працівників, поліпшує умови праці, що в свою чергу, призводить до підвищення продуктивності праці, зниження кількості нещасних випадків, запобігання  виходу з ладу обладнання та інших нештатних ситуацій.

Покращення умов праці та її безпека призводить до зменшення виробничого травматизму, професійних хвороб, що зберігає здоров’я працівників та одночасно призводить до зменшення затрат на оплату пільг та компенсацій, на оплату наслідків такої роботи, на лікування, перепідготовку працівників виробництва в зв’язку зі зміною кадрів через причини, що пов’язані з умовами праці.
\subsection{Аналіз робочого місця}
Приміщення, в якому розроблювалася система буде розташоване на другому поверсі п'ятиповерхового будинку та розраховано на чотири робочих місця. Схема приміщення представлена нижче:

\begin{figure}[H]
	\centering
	\caption{Схематичний план приміщення}
	\includegraphics[width=150pt]{safety_room_plan}
	\label{fig:safety_room_plan}
\end{figure}

\begin{tabular}{l}
	Довжина приміщення: 6м;\\
	Ширина приміщення: 4.5м;\\
	Висота приміщення: 3.5м;\\
	Колір підлоги – темний, стін та стелі – світлий.\\
	Коефіцієнти відображення:
\end{tabular}

\begin{equation}
	\rho_\textup{стелі}=70\%,
	\rho_\textup{стін}=50\%,
	\rho_\textup{підлоги}=30\%
\end{equation}

\begin{tabular}{l}
	В приміщенні є одне вікно розмірами: 1.5 м шириною і 2 м висотою.\\
	Площа даного приміщення: $S=6\textup{м}*4.5\textup{м}=27\textup{м}^2$ \\
	Об'єм даного приміщення: $V=S*h=27\textup{м}^2*3.5\textup{м}=94.5\textup{м}^3$\\
	Таким чином, на одне робоче місце надано: $S=6.75\textup{м}^2, V=23.625\textup{м}^3$\\
	Порівняємо розрахункові значення з нормативними у таблиці:
\end{tabular}

\begin{table}[H]
	\centering
	\caption{Розрахункові та нормативні значення площі та об'єму приміщення з розрахунку на одне рабоче місце}
	\begin{tabular}{| l | l | l |}
		\hline
		Параметр приміщення & Нормативний & Розрахунковий\\\hline
		Площа, $\textup{м}^2$ & 6 і більше & 6.75\\\hline
		Об'єм, $\textup{м}^3$ & 20 і більше & 23.625\\\hline
	\end{tabular}
\end{table}

Отже, можна зробити висновок, що площа та об’єм робочого місця відповідає нормам НПАОП 00.0-1.28-10.
\subsection{Аналіз шкідливих і небезпечних факторів}
\subsubsection{Мікроклімат}
Санітарні норми мікроклімату виробничих приміщень в цьому розділі описані згідно ДСН 3.3.6.042-99. Робота, виконувана в даному приміщенні відноситься до категорії робіт – «Легка 1б». У приміщеннях з використанням обчислювальної техніки рекомендується застосування тільки оптимальних значень показників мікроклімату. Нижче приведено видповідні санітарні вимоги до мікроклімату в приміщенні, що повинні дотримуватися.

\begin{table}[H]
	\centering
	\caption{Оптимальні значення параметрів мікроклімату для категорії робіт ``Легка-1б''}
	\begin{tabular}{| l | r | r | r | }
		\hline
		Пора року & Температура, $^oC$ & Вологість, \% & Швидкість повітря, м/с \\\hline
		Тепла & 22-24 & 40-60 & 0,1 \\\hline
		Холодна	& 21-23 & 40-60 & 0,1 \\\hline
	\end{tabular}
	\label{tab:micro-climate}
\end{table}

У приміщенні встановлені батареї центрального водяного опалення, що включається в холодний період року. У теплу пору працює, система кондиціонування, що складається з кондиціонера спліт-системи Кондиціонер DELFA ADW-07C з потужністю 1100 Вт.
\subsubsection{Освітлення}
В приміщеннях для роботи з ЕОМ повинне використовуватися як природне так і штучне освітлення. Природне освітлення забезпечує вікно, загальна площа якого складає $3\textup{м}^2$. Воно являється боковим та одностороннім.

Нормоване значення КПО, яке має забезпечувати природне освітлення розраховується за формулою:

\begin{equation}
	e_\textup{н}=\frac{S_\textup{вік}}{S_n}=\frac{3}{27}=0.11,
\end{equation}
 
де $e_\textup{н}$ -- значення КПО; $S_\textup{вік}$ -- загальна площа вікна, $\textup{м}^2$; $S_n$ -- площа підлоги, $\textup{м}^2$. 

Отримане значення $0.11$ менше ніж встановлено нормами (КПО має бути не меншим за $0.15$), тобто природного освітлення не вистачає для нормальної роботи. Тому потрибно використовувати штучне освітлення.

Штучне освітлення в приміщеннях з робочими місцями, обладнаними ВДТ ЕОМ та ПЕОМ, має здійснюватися системою загального рівномірного освітлення. У якості джерел світла для штучного освітлення мають застосовуватись переважно люмінесцентні лампи типу ЛБ, потужністю 20Вт. Для загального освітлення слід застосовувати 2 світильники серії ЛПО, розташовані у 2 ряди. Один світильник містить 2 лампи, кожна з яких має світловий потік 1060 лм. Нормативна освітленість 300-400 лк., згідно ДБН 2006. Штучне освітлення кімнати створює освітленість 340 лк, що задовольняє стандарту.
\subsubsection{Шум}
Основним джерелом шуму є системний блок комп’ютера, який містить такі компоненти як: жорсткий диск та кулер.

Таким чином у приміщенні мають місце шуми механічного і аеродинамічного походження. Шум, що створюється, умовно можна віднести до постійного.

Згідно з ДСН 3.3.6.037-99 допустимий шум на постійних робочих місцях користувача складає до 50 дБА. Орієнтовні еквівалентні рівні звукового тиску джерел шуму, що діють на користувача на його робочому місці, представлені в табл. \ref{tab:sound_levels}.

\begin{table}[H]
	\centering
	\caption{Рівні звукового тиску від різних джерел}
	\begin{tabular}{| l | r | }
		\hline
		Джерело шуму & Рівень шуму, дБА \\\hline
		Жорсткий диск & 26 \\\hline
		Кулер & 28 \\\hline
	\end{tabular}
	\label{tab:sound_levels}
\end{table}

Розрахуємо середній рівень шуму на робочому місці користувача при роботі всієї вказаної техніки.

Рівень шуму, що виникає від декількох некогерентних джерел, що працюють одночасно, підраховується на підставі принципу енергетичного підсумовування рівня інтенсивності окремих джерел:

\begin{equation}
	L = 10\lg\sum10^{0.1L_i},
\end{equation}
де $L_i$ -- рівень звукового тиску і-того джерела.

Підставивши значення рівня звукового тиску для кожного виду устаткування у формулу, отримаємо: 

\begin{equation}
	L = 10\lg(4*10^{0.1*26}+4*10^{0.1*28})=36\textup{дБ}.
\end{equation}

Розраховане значення рівня шуму не перевищує гранично допустимого рівня шуму для робочого місця користувача (50 дБА), тобто спеціальні заходи по зниженню рівня шуму не потрібні.

Таким чином, робота з системою, розробленою в дипломній роботі, являється безпечною і не потребує додаткових улаштувань для зниження шуму, окрім загальних методів ізоляції від зовнішнього шуму. Для цього застосовуються  спеціальні віконні профілі та звукоізоляція зовнішніх стін плитами зі звукоізоляційними наповнювачами.
\subsection{Електромагнітні випромінювання}
Більшість учених вважає, що як короткочасна, так і тривала дія всіх видів випромінювання від екрану монітора не небезпечна для здоров'я людини. Проте вичерпних даних щодо небезпеки дії випромінювання від моніторів на людей, що працюють з комп'ютерами не існує і дослідження в цьому напрямі продовжуються.

Допустимі значення параметрів не іонізуючих електромагнітних випромінювань від монітора комп'ютера представлені в табл. \ref{tab:x-ray}.

Максимальний рівень рентгенівського випромінювання на робочому місці оператора комп'ютера звичайно не перевищує 10мкбэр/ч, а інтенсивність ультрафіолетового і інфрачервоного випромінювань від екрану монітора лежить в межах 10-100мВт/м$^2$.

\begin{table}[H]
	\centering
	\caption{Допустимі значення параметрів не іонізуючих електромагнітних випромінювань}
	\begin{tabular}{| l | r | }
		\hline
		Найменування параметра & Допустимі \\\hline
	    Напруженість електричної складової електромагнітного & \\
		поля на відстані 50см від поверхні відеомонітора & 10В/м \\\hline
	    Напруженість магнітної складової електромагнітного & \\
		поля на відстані 50см від поверхні відеомонітора & 0.3А/м \\\hline
		Напруженість поля не повинна перевищувати:  & \\
		для дорослих користувачів & 20кВ/м \\
		для дітей дошкільних установ і що вчаться в & 15кВ/м \\
		середніх спеціальних і вищих учбових закладів & \\\hline
	\end{tabular}
	\label{tab:x-ray}
\end{table}

Для зниження дії цих видів випромінювання рекомендується застосовувати монітори із зниженим рівнем випромінювання (MPR-II, TCO-92, TCO-99, TCO-03), а також дотримувати регламентовані режими праці і відпочинку.
\subsection{Електробезпека}
Згідно з ДНАОП 0.00-1.3.1-99 робоче місце підпадає під категорію без підвищеної небезпеки.

Електроустаткування належить до приладів до 1000 В. Устаткування, що використовується, відповідно до ПУЄ належить до устаткування класів 0, 0І, і І за електрозахистом.

Оцінка небезпеки дотику до струмових частин відноситься до визначення сили струму, що протікає через тіло людини, і порівняння його із допустимим значенням відповідно до ГОСТ 12.1.038-88.

Лінія електромережі для живлення персональних комп'ютерів, їх периферійних пристроїв (принтер) виконується як окрема групова три-провідна мережа, шляхом прокладання фазового, нульового робочого та нульового захисного провідників. Нульовий захисний провідник використовується для заземлення електроприладів. Провід мідний, ізоляція має бути закритою, марки ПУНП, перерізом не менше 2,5х2мм на жилу. Частота струму не має перевищувати значення 50 Гц.

При виконанні розрахунків для дипломного проекту використовувався персональний комп'ютер - І і II клас захисту, що живиться напругою 220 В. Для правильного визначення необхідних засобів та заходів захисту від ураження електричним струмом необхідно знати допустимі значення напруг доторкання та струмів, що проходять через тіло людини.

Гранично допустимі значення напруги доторкання та сили струму для нормального (безаварійного) та аварійного режимів електроустановок при проходженні струму через тіло людини по шляху ``рука – рука'' чи ``рука – ноги'' регламентуються ГОСТ 12.1.038-88 (табл. \ref{tab:current}).

\begin{table}[H]
	\centering
	\caption{Граничнодопустимі значення напруги доторкання та сили струму, що проходить через тіло людини при нормальному режимі електроустановки}
	\begin{tabular}{| l | r | r |}
		\hline
		Вид струму & В (не більше) & мА (не більше) \\\hline
		Змінний, 50Гц & 2 & 0.3 \\\hline
		Змінний, 400Гц & 3 & 0.4 \\\hline
		Постійний & 8 & 1.0 \\\hline
	\end{tabular}
	\label{tab:current}
\end{table}
\subsection{Пожежна безпека}
Згідно з НАПБ Б.03.002-2007 таке приміщення відноситься до категорії    В–пожежонебезпечна. При нормальному режимі роботи можливість виникнення пожежі мінімальна. Можливість виникнення вибухів повністю відсутня. Можливими причинами загоряння можуть бути пошкодження та замикання в електромережі та електрообладнанні, а також порушення правил безпеки при роботі з обладнанням.

На робочому місці наявні наступні пожежонебезпеці матеріали: папір, пластик, віконні рами, дерев’яні шафи, корпуси техніки, меблі. Робоче приміщення повинно бути обладнане двома вуглекислотними вогнегасниками ВВК-5 з розрахунку два вогнегасника на приміщення  до   25 кв.м  включно, що задовольняє НАПБ Б.03.002-2007. Для захисту від блискавки будівля обладнана блискавковідводом стрижневого типу.

В приміщенні посередині стелі має бути встановлений один димовий пожежний сповіщувач СПД-3 відповідно до ДБН В.1.1.-7-2002 – з розрахунку один на висоту до 3,5 м та загальною площею не більше ніж 86 м$^2$.

Технічні заходи щодо зниження пожежної небезпеки на підприємстві:
\begin{itemize}
	\item застосування засобів пожежогасіння;
	\item використання засобів пожежної сигналізації;
	\item проведення пожежотехнічних обстежень;
	\item використання знаків.
\end{itemize}

    \conclusion

    Деякі висновки.

    \begin{thebibliography}
\bibitem{Landafshitz}
Ландау~Л.~Д., Лившиц~Е.~М. Теоретическая физика: Учеб. пособ.: Для вузов. В 10 т. Т. VI. Гидродинамика. --- 5-е изд., стереот. --- М.:~ФИЗМАТЛИТ. --- 2001. --- 736~с. --- ISBN5-9221-0121-8 (T. VI).

\bibitem{Sokolofsky}
Sokolofsky~S.~A., Jirka~J.~A. CVEN 489-501: Special Topics on Mixing and Transport in the Environment. Engineering -- Lectures. --- 5th edition. --- Texas A \& M University. --- 2005. --- 184~pp.

\bibitem{Bejan}
Adrian~Bejan. Convection Heat Transfer. --- 4th edition. --- Wiley. --- 2013. --- 696~pp.

\bibitem{Lewis}
Lewis~R.~W. Fundamentals of the Finite Element Method for Heat and Fluid Flow. / Ronald~W.~Lewis, Perumal Nithiarasu, Kankanhally N. Seetharamu. --- Wiley. --- 2004. --- 356~pp.

\bibitem{Vlasova}
Власова~Е.~А. Приближенные методы математической физики. / Е.~А.~Власова, В.~С.~Зарубин, Г.~Н.~Кувырикин. --- М.:~Изд-во МГТУ им.~Н.~Э.~Баумана. --- 2001. --- 700~с.

\bibitem{Brebbia}
Бреббия~К. Метод граничных элементов: Пер. с англ. / Бреббия~К., Теллес~Ж., Вроубел~Л. --- М.:~Мир. --- 1987. --- 524~с.

\end{thebibliography}


\end{document}

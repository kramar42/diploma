\documentclass[a4paper,14pt]{report}
\begin{document}
\section{Постановка задачі}
Було вибрано реалізувати бібліотеку, яка б містила деякі з запропонованих методів пошуку, а саме:
\begin{itemize}
	\item Мінімаксний пошук по дереву
	\item Порівняння зі зразком
	\item Метод Монте-Карло
\end{itemize}
Ці методи були вибрані тому, що вони самі по собі не залежать від гри, а тількі від внутрішнього представлення цієї гри. Також цим методам, на відміну від інших, не треба ``пояснювати'' правила го, вони можут працювати окремо і бути використані у іншому модулі для аналогічного пошуку у подібних структурах даних інших ігор.

Створити бібліотеку було вибрано тому, що це гарний спосіб інкапсулювати декілька реалізацій методу пошуку у структурах даних гри го. Також у такий спосіб інші програми або модулі можуть використовувати ці реалізації, більш піклуючись про інші аспекти гри.
\end{document}
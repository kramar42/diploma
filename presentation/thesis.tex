\documentclass{diploma}

\begin{document}
\maketitlepage{Борисенко Павло Борисович}{КМ-02}{ст. викл. Любашенко Н.Д.}{доцент, к. ф.-м. н. Шубенкова І.А.}
{Моделювання процесів конвекції-дифузії з переважанням дифузії}

\assigment{
	StudentName={Борисенку Павлу Борисовичу},
	ThesisName={\invcommas{Моделювання процесів конвекції-дифузії з переважанням дифузії}},
	AdvisorName={ст. викладач Любашенко Наталія Дмитрівна},
	Order={\invcommas{28}~травня~2014~р.~\No~995-C},
	ApplicationDate={\invcommas{15}~червня~2014~р.},
	InputData={\begin{itemize}
			\item алгоритм методу граничних елементів;
			\item дані задачі дифузії.
		\end{itemize}},
	Contents={\begin{itemize}
			\item аналіз існуючих методів вирішення задачі;
			\item вибір омтимального методу;
			\item програмна реалізація;
			\item порівняння результатів.
		\end{itemize}},
	Graphics={\begin{itemize}
			\item блок-схеми алгоритмів;
			\item знімки екранних форм.
		\end{itemize}},
	AssigmentDate={\invcommas{01}~жовтня~2013~р.},
	Calendar={& & & \\},
	StudentPIB={Борисенко П.Б.},
	AdvisorPIB={Любашенко Н.Д.}
}

\annotation{Анотація}
Дипломна робота присвячена розробці математичних та програмних засобів для моделювання процесів конвекції-дифузії з переважанням дифузії.

У рамках роботи проведено аналіз підходів до математичного моделювання процесів конвекції-дифузії, що виникають в прикладних задачах фізики твердого тіла, гідро- та аеродинаміки, біологічних та медичних дослідженнях. Обрано математичну модель задачі з урахуванням переважання дифузних процесів.

Розглянуто способи чисельного моделювання процесів конвекції-дифузії з переважанням дифузії на основі обраної математичної моделі та обгрунтовано використання методу граничних елементів.

Розроблено програмне забезпечення, що реалізовує обрані методи моделювання дифузних процесів.

Ключові слова: процес конвекції-дифузії, дифузія, метод граничних елементів, метод колокації.

\annotation{Abstract}
In this thesis we consider the development of mathematical and software tools for simulation of convection-diffusion processes with diffusion prevalence.

In this paper we perform the analysis of theorethical approaches to convection-diffusion simulation that can be applied to the different problems in solid state physics, hydrodynamics, aerodynamics, biology and medical studies. We choose the mathematical model of this problem subject to prevalence of diffusion.

We discuss the methods of numerical modeling of convection-diffusion processes with diffusion prevalence based on choosed mathematical model and substantiate using of boundary elements method.

We describe software tools for simulation of diffusion processes developed as part of this thesis.

Keywords: convection-diffusion process, diffusion, boundary elements method, collocation method.

\annotation{Аннотация}
Дипломная работа посвящена разработке математических и программных средств для моделирования процессов конвекции-диффузии с преобладанием диффузии.

В рамках работы проведено анализ подходов к математическому моделированию процессов конвекции-диффузии, которые возникают в прикладных задачах физики твердого тела, гидро- и аеродинамики, биологических и медициских исследованиях. Выбрано матматическую модель задачи с учетом преобладания диффузных процессов.

Рассмотрено способы численного моделирования процессов конвекции-диффузии с преобладанием диффузии на основании выбраной математической модели и обосновано использование метода граничных элементов.

Разработано програмное обеспечение, реализующее выбраный метод моделирования диффузных процессов.

Ключевые слова: процесс конвекции-диффузии, диффузия, метод граничных элементов, метод коллокации.

\tableofcontents

\intro
Задачі моделювання процесів конвекції-дифузії часто виникають в різноманітних областях прикладних досліджень, особливо таких, що стосуються розподілу температур чи переміщення часток речовин: фізиці твердого тіла, гідро- та аеродинаміці, медичних та хімічних дослідженнях.

Окрім того, задачі конвекції-дифузії тісно пов'язані зі схожими задачами з інших областей: статистичної механіки (рівняння Фоккера--Планка), фінансової математики (рівняння Блека--Скоулза), гідродинаміки (рівняння Нав'є--Стокса) тощо. Таким чином, моделювання і аналіз таких задач має велике практичне значення.

Такі задачі рідко мають аналітичне вирішення, тому виникає потреба у чисельному їх розв'язанні, що в сучасних умовах породжує необхідність у виборі швидкого, точного та ефективного матеамтичного методу чисельного вирішення задачі та розробці на його основі алгоритму, що може бути використаний для чисельного моделювання дифузних процесів різної природи на ЕОМ.

\chapter{Мета дослідження та постановка задачі}
Метою виконання дипломної роботи є розробка математичного та програмного забезпечення для моделювання процесів конвеуції-дифузії з переважанням дифузії різної природи.

Предметом дослідження є різноманітні дифузійні процеси у фізиці твердого тіла, гідро- та аеродинаміці, біологічних, хімічних та медичних дослідженнях. Об'єктом дослідження дипломної роботи є методи чисельного моделювання таких процесів.

Математична модель має реалістично описувати процеси перенесення та взаємопроникнення тепла чи часток речовини, що відбуваються в ході моделюованого дифузійного процесу. Схема чисельного моделювання має забезпечувати точність отримуваного результату.

Розроблюване програмне забезпечення повинно з одного боку реалізовувати точний та швидкий алгоритм обчислень, будучи при цьому невибагливим до обчислювальних ресурсів. Окрім того воно має надавати засоби візуалізації результатів моделювання.

Таким чином, задачами, що мають бути вирішені в даній дипломній роботі є:
\begin{enumerate}
	\item побудувати математичну модель процесу конвекції-дифузії з переважанянм дифузії;
	\item шляхом аналізу існуючих підходів до вирішення задач дифузії обрати метод чисельного моделювання процесу на основі обраної моделі;
	\item реалізувати програмне забезпечення для моделювання процесу за допомогою обраного чисельного методу.
\end{enumerate}

\chapter{Аналіз існуючих методів моделювання процесів конвекції-дифузії}
\section{Вибір математичної моделі}
Процес конвекції-дифузії --- це процес переносу та взаємопроникнення часток речовини, що зазвичай приводить до вирівнювання їх концентрацій по всьому об'єму. Зазвичай під дифузією розуміють процеси, що супроводжуються переносом речовини, однак іноді дифузійними називають також інші процеси перенесення: теплопровідність, в'язке тертя тощо \cite{Landafshitz}.

Типовою математичною моделлю процесу конвекції-дифузії виступає диференційне рівняння переносу скалярної величини у просторі~\cite{Sokolofsky}:
\begin{equation} \label{eq:ConvDiffGeneral}
	\frac{\partial c}{\partial t} = \nabla \cdot (D \nabla c) - \nabla \cdot (\vec{v} c) + R,
\end{equation}
де:
\begin{itemize}
	\item $c$ --- змінна, що відображає перенос (концентрація речовни, температура тощо);
	\item $D$ --- коефіцієнт дифузії, наприклад, коефіцієнт теплопровідності для переносу температури;
	\item $\vec{v}$ --- середня швидкість переносу;
	\item $R$ --- змінна, що описує \invcommas{стоки} та \invcommas{джерела} велични $c$. Наприклад, для хімічної реакції $R>0$ може означати, що в результаті реакції утворюється речовина, а $R<0$ --- що речовина розкладається.
\end{itemize}

У більшості випадків для спрощення розв'язання рівняння~\ref{eq:ConvDiffGeneral} висувається прпущення про сталість коефіцієнту дифузії, відсутність \invcommas{стоків} та \invcommas{джерел} і вважається, що потік нестисний. В такому випадку рівняння набуває вигляду~\cite{Bejan}
\begin{equation} \label{eq:ConvDiffSimple}
	\frac{\partial c}{\partial t} = D \nabla^2 c - \vec{v} \cdot \nabla c.
\end{equation}

Якщо знехтувати переносом, то отримаємо параболічне диференційне рівняння в частинних похідних, що називається рівнянням теплопровідності:
\begin{equation} \label{eq:Heat}
	\frac{\partial c}{\partial t} = D \nabla^2 c.
\end{equation}

Рівняння вигляду~\ref{eq:ConvDiffGeneral},~\ref{eq:ConvDiffSimple} та~\ref{eq:Heat} часто зустрічаються у різних областях фізики: вони є ідентичними до рівнянь Фоккера--Планка у статистичній механіці, рівняння дрейфу-переносу у фізиці напівпровідників, тісно пов'язані з рівняннями Блека--Скоулза та Нав'є--Стокса.

\section{Вибір методу розв'язаня задачі}
Найчастіше розглядається крайова задача для рівнянь вигляду~\ref{eq:ConvDiffSimple} та~\ref{eq:Heat}. Однак розглянуті нижче методи підходять і для рівнянь загального вигляду~\ref{eq:ConvDiffGeneral}.

Враховуючи форму цих рівнянь, проаналізуємо методи чисельного розв'язку для рівняння першого порядку
\begin{equation} \label{eq:FirstOrder}
	\frac{\partial c}{\partial t} = D \frac{\partial^2 c}{\partial x^2} - v\nabla \frac{\partial c}{\partial x} + R.
\end{equation}

Реультати можна легко узагальнити на випадок 2-хвимірної і 3-хвимірної координатної системи.

Найпоширенішим методом розв'язання рівнянь конвекції-дифузії є методи скінченних та граничних елементів.

\subsection{Метод скінченних різниць}
Метод скінченних різниць --- чисельний методи розв'язку інтегро-диференціальних рівнянь алгебри, диференціального, інтегрального числення, заснований на заміні диференціальних операторів різницевими операторами, інтегралів --- сумами, а функцій неперервного аргументу --- функціями дискретного аргументу. Така заміна приводить до системи, взагалі кажучи, нелінійних алгебраїчних рівнянь, які зрештою зводяться до лінійної системи деяким ітераційним методом.

\subsubsection{Явна схема}
Для рівняння~\ref{eq:FirstOrder} явна схема має вигляд
\begin{equation} \label{eq:ExplicitScheme}
	\begin{array}{l}
	\frac{c_i ^f - c_i ^{f-1}}{\Delta t} = D \frac{c_{i-1} ^{f-1} - 2c_i ^{f-1} + c_{i+1} ^{f-1}}{h^2} - v \frac{c_{i+1} ^{f-1} - c_{i-1} ^{f-1}}{2h} + R_i ^{f-1}, \\
	c_i ^f = \left(1-\frac{2D\Delta t}{h^2}\right) c_i ^{f-1} + \left(\frac{D \Delta t}{h^2} + \frac{v \Delta t}{2h}\right) c_{i-1} ^{f-1} + \left(\frac{D \Delta t}{h^2} - \frac{v \Delta t}{2h}\right) c_{i+1} ^{f-1} + R_i ^{f-1} \Delta t,
	\end{array}
\end{equation}
де $\Delta t = t^{f} - t^{f-1}$, $h$ --- сталий крок сітки.

В цій схемі нові значення повністю залежать від попередніх значень (тобто від початкових умов).

Критерій стабільності:
\begin{equation} \label{eq:ExplicitSchemeStability}
	h < \frac{2D}{v}, ~~~~ \Delta t < \frac{h^2}{2D}.
\end{equation}

Ці нерівності накладають серйозні обмеження на явну схему. Цей метод не рекомендується застосовувати, оскільки максимально можливий крок в часі зменшується з квадратом $h$.

\subsubsection{Неявна схема}
В неявній схемі величина залежна від нового значення часу $t+\Delta t$. Після застосування неявної схеми усі коефіцієнти виходять додатніми, що робить її стійкою для довільного кроку в часі. Ця схема зазвичай застосовується через її стійкість і точність. Однак її недоліками є складність обчислень та помилки переповнення, що можуть виникати при великих значеннях кроку $\Delta t$.

\subsubsection{Схема Кранка--Ніколсон}
У схемі Кранка--Ніколсон величина залежна однаково від $t$ і від $t+\Delta t$. Цей метод другого порядку зазвичай застосовується для задач дифузії:
\begin{equation} \label{eq:CrankNikolsonScheme}
		c^{f+1}_j = c^{f}_j + \frac{D\Delta t}{2h^2} (c^{f+1}_{j+1} - 2c^{f+1}_{j} + c^{f+1}_{j-1})	+ \frac{D\Delta t}{2h^2} (c^{f}_{j+1} - 2c^{f}_{j} + c^{f}_{j-1}).
\end{equation}

Його критерій стабільності
\begin{equation} \label{eq:CrankNikolsonStability}
	\Delta t < \frac{h^2}{D}
\end{equation}
менш обмежуючий, ніж у явного методу.

Схема Кранка--Ніколсон заснована на центральній різницевій схемі, а тому є методом другого порядку з точністю по часу.

\subsection{Метод скінченних елементів}
Метод скінчених елементів --- це числова техніка знаходження розв'язків інтегральних та часткових диференціальних рівнянь. Процес розв'язання побудований або на повному усуненні диференціального рівняння для стаціонарних задач, або на розкладі часткових диференціальних рівнянь в апроксимуючу систему звичайних диференціальних рівнянь, які потім розв'язуються використанням якої-небудь стандартної техніки, такої як метод Ейлера, Рунге-Кутти тощо.

Проблеми, що виникають у методі скінченнх елементів, пов'язані з вирішенням конвекційної частини рівняння~\ref{eq:FirstOrder}. Якщо число Пекле перевищує деяку китичну величину, у середовищі виникають осциляційні процеси. У методі скінченних різнииць ці осциляції зменшуються за рахунок сімейства дискретизаційних схем, наприклад схемами з різницями проти потоку. В цьому методі це досягається за рахунок зміни базових формуючих функцій.

Якщо рівняння залежить від часу, скінченно-різницева схема має відповідний метод скінченних елементів (метод Гальоркіна). Характеристичний метод Гальоркіна використовує неявну схему. Для скалярних змінних ці два методи співпадають~\cite{Lewis}.

Найважливішими перевагами методу скінченних елементів є:
\begin{itemize}
	\item скінченними елементами є прості області (прямі лінії, трикутники, прямокутники, піраміди, призми); таким чином, даним методом можна апроксимувати тіла із складною формою країв;
	\item розміри елементів можуть бути змінними; це дозволяє збільшувати чи зменшувати елементи сітки;
	\item за допомогою цього методу легко розглянути граничні умови з розривним поверхневим навантаженням, а також змішані граничні умови;
	\item алгоритм методу скінченних елементів дозволяє створити загальні програми для розв'язку завдань різного класу;
	\item завдання зводиться до розв'язку системи рівнянь великої розмірності. Проте гарна обумовленість системи розв'язних рівнянь  дозволяє отримувати досить точні розв'язки для систем розмірністю 5-10 мільйонів і більше.
\end{itemize}

Головний недолік цього методу полягає у потребах великого обсягу пам'яті ЕОМ і високої швидкості розрахунку.

\subsection{Метод граничних елементів}
Метод граничних елементів можна розглядати як модифікацію методу скінченних елементів для апроксимації шуканих функцій, але не в області вирішення задачі, а на її границі. Це дозволяє понизити розмірність задачі. В теоретичнй основі методу лежить перехід від частинних диференційних рівнянь до інтегральних рівнянь. Формулювання задачі в методі граничних елементів включає інтеграли шуканих функцій та їх похідних, що обчислюються лише по границі області~\cite{Vlasova}.

Для одновимірного випадку з~\ref{eq:FirstOrder} без врахування конвекційної частини відповідне інтегральне рівняння матиме вигляд~\cite{Brebbia}:
\begin{equation} \label{eq:Brebbia}
	\begin{array}{l}
	c(\xi, t_F) + D \int^{t_F}_{t_0} \int_\Gamma c(x, t)\frac{\partial c^*}{\partial x}(\xi,x,t_F,t) d\Gamma(x) dt = \\
	D \int^{t_F}_{t_0} \int_\Gamma \frac{\partial c}{\partial x}(x, t)c^*(\xi,x,t_F,t) d\Gamma(x) dt + \int_\Omega c_0(x, t_0)c^*(\xi,x,t_F,t_0)d\Omega(x),
	\end{array}
\end{equation}
де:
\begin{itemize}
	\item $\xi$ --- точка, що належать границі області;
	\item $t_F$ --- кінцевий час;
	\item $\Gamma$ --- границя області вирішення задачі;
	\item $\Omega$ --- область вирішення задачі;
	\item $c^*$ та $\frac{\partial c^*}{\partial x}$ --- фундаментальне рішення та його похідна;
	\item $c_0(x, t_0)$ --- початкові умови.
\end{itemize}

Для отримання чисельного рішення можна скористатися двома схемами: перша розглядає кожен крок по часу як нову задачу, через що в кінці кожного кроку рахується значення функції $c$ в досить великій кількості внутрішніх точок; друга завжди починає інтегрування по часу з моменту $t_0$ і таким чином ункає необхідності обчислення значень функції у внутрішніх точках при збільшенні кількості проміжних кроків.

\chapter{Математичне забезпечення}


\chapter{Програмне забезпечення}
\begin{equation} \label{eq:testTask1}
	\begin{aligned}
		& u_0(x, 0) = 20, \forall x; \\
		& u (0, t) = 300, t > 0; \\
		& u (l, t) = 100, t > 0.
	\end{aligned}
\end{equation}

\chapter{Охорона праці}
Законодавчі та інші нормативно-правові акти з охорони праці встановлюють, регламентують та регулюють державні вимоги щодо забезпечення безпечних і нешкідливих умов праці, сприяють створенню та ефективному функціонуванню чіткої системи управління охороною праці на підприємстві, в галузі, в регіоні і державі в цілому, забезпеченню на кожному робочому місці безпечних і нешкідливих умов праці, встановлення правила безпечного виконання робіт і поведінки працівників на території підприємства, у виробничих приміщеннях, на будівельних майданчиках, на інших робочих місцях.

Ефективність використання програмного продукту, залежить також від умов, у яких доведеться працювати його користувачам, що має бути враховано вже на етапі проектування. Окрім того, необхідно враховувати специфіку організації роботи з візуальними дисплейними терміналами електронно-обчислювальних машин.

У даному розділі спроектовано приміщення для роботи двох операторів ЕОМ, що забезпечить безпечні та нешкідливі умови праці. Проектоване приміщення знаходиться на четвертому поверсі офісної будівлі.

\section{Характеристика робочого місця}
Розглядається офісне приміщення призначене для роботи двох операторів ЕОМ. Офісне приміщення розташоване на четвертому поверсі. Приміщення має два робочих місця, обладнаних персональними комп’ютерами з периферією. Схема розташування робочих місць подана на рис.~\ref{workspace}.

Розміри приміщення:
\begin{itemize}
	\item ширина --- 4~м;
	\item довжина --- 3,5~м;
	\item висота --- 3,5~м.
\end{itemize}

Площа приміщення становить 14~м$^2$; об’єм --- 49~м$^3$. На кожного працівника припадає по 7~м$^2$ площі та 24,5~м$^3$ об’єму, що відповідає нормам, встановленим ДСанПіН~3.3.2-007-98.

У приміщенні застосовується комбіноване освітлення. Природне освітлення забезпечується вікном площею 7,9~м$^2$. Штучне освітлення забезпечується лампами ЛБ40-1 згідно ДБН~В.2.5-28-2006.

Матеріали стін характеризуються коефіцієнтом відбиття 0,8. Матеріали стелі характеризуються коефіцієнтом відбиття 0,5.

\begin{figure}[!h]
	\centering
	\includegraphics[scale=0.7]{workspace.png}
	\caption{План робочого приміщення}
	\label{workspace}
\end{figure}

Підлога рівна, вкрита антистатичним ламінатом з неслизьким покриттям. У приміщенні функціонують системи опалення та кондиціонування повітря згідно вимог ДСН~3.3.6.042-99. Щоденно проводиться вологе прибирання.

\section{Шкідливі та небезпечні фактори}

\subsection{Аналіз мікроклімату}
Робота операторів ЕОМ належить до категорії легких робіт, Іа. Оптимальні згідно ДСН~3.3.6.042-99 параметри мікроклімату для зазначеної категорії робіт наведено у таблиці~\ref{microclimate}.

\begin{table}{|c|c|c|}{Оптимальні значення параметрів мікроклімату}{microclimate}
{\hline
Період року & Параметр мікроклімату & Оптимальне значення \\
\hline}
\multirow{3}{*}{Холодний}	& Температура повітря в приміщенні	& 22...24~$^\circ$C \\
							& Відносна вологість				& 40...60~\% \\
							& Швидкість руху повітря			& до~0,1~м/с \\
\hline
\multirow{3}{*}{Теплий}	& Температура повітря в приміщенні	& 23...25~$^\circ$C \\
						& Відносна вологість				& 40...60~\% \\
						& Швидкість руху повітря			& 0,1~м/с \\
\end{table}

Дотримання оптимальних значень параметрів мікроклімату забезпечується використанням систем опалення та кондиціонування.

Для внутрішнього теплопостачання використовується водяна двотрубна система опалення з радіаторами.

Згідно ДБН~В.2.5-67-2013 об’ємні витрати свіжого повітря, що подається в приміщення мають становити не менше 30 м$^2$ на людину, тому використовується загальнообмінна приливно-витяжна система кондиціонування у вигляді спліт-системи з настінним монтажем та пониженим рівнем шуму.

\subsection{Шум}
Вікна приміщення виходять на внутрішній майданчик будівлі, що віддалений автомобільних і залізничних шляхів. Приміщення не призначене для прийому відвідувачів, тому основним джерелом шуму є комп’ютерна та офісна техніка.

Оскільки використовувана техніка є однотипною, можемо обчислити сумарний рівень сили шуму за формулою~\ref{noize}.
\begin{equation} \label{noize}
	L = L_1 + 10 \lg n
\end{equation}

У приміщенні розташовано два комп’ютери, сила шуму кожного з яких у робочому режимі станови 30~дБА. Звідси, скориставшись формулою~\ref{noize}, отримуємо загальний рівень сили шуму $L = 30 + 10 \lg 2 = 30 + 3 =\\ 33$~дБА, що менше норми в 50~дБА і відповідає вимогам ДСН~3.3.6.037-99.

\subsection{Виробниче освітлення}
Робота оператора ЕОМ належить до робіт середньої точністю, IV розряд зорових робіт.

Згідно ДБН~13.2.5-28-2006, робоче приміщення, як і будь-яке інше виробниче приміщення, повинне мати систему комбінованого освітлення. Природне освітлення представлене системою однобічного бічного освітлення, а штучне освітлення здійснюється за допомогою системи загального рівномірного освітлення.

Мінімальна освітленість для IІ розряду зорових робіт складає 300-500~лк. Проведемо розрахунок параметрів системи штучного освітлення. Для визначення потрібної кількості світильників обчислимо світловий потік, що падає на робочу поверхню за формулою~\ref{light}.
\begin{equation} \label{light}
	F = \frac{EKSZ}{\eta}
\end{equation}

Скориставшись формулою~\ref{light}, отримуємо величину світлового потоку $F = \frac{300 \cdot 1,5 \cdot 14 \cdot 1,1}{0,22} = 31500$~лм. Для освітлення використовуються 4 світильники типу ШОД з люмінесцентними лампами типу ЛБ40-1, світловий потік яких $F_\text{л} = 4320$~лм. Таким чином, необхідно використати $n = F/F_\text{л} = 31500/4320 = 8$~ламп. Кожен світильник комплектується двома лампами.

\section{Електробезпека}
Згідно правил улаштування електроустановок робоче приміщення належить до категорії з умовами без підвищеної небезпеки ураження електричним струмом, оскільки у приміщенні відсутні умови, що створюють підвищену та особливу небезпеку.

У приміщенні розміщено два персональних комп’ютери, що живляться від однофазової мережі з напругою 220~В із заземленою нейтраллю.

Основним джерелом небезпеки ураження людини електричним струмом є струмопровідні частини персональних комп’ютерів. Для унеможливлення ненавмисного контакту з провідниками струму у персональному комп’ютері вони вміщені у непровідний пластико-металевий корпус, що відповідає першому класу захисту.

Крім того, джерелами ураження електричним струмом можуть бути місця порушення ізоляції електромережі при короткому замиканні чи у результаті недотримання правил електротехнічної безпеки. Для уникнення цих ситуацій електропроводка має подвійну ізоляцію та розведена у антистатичних тримачах всередині стін.

Для попередження коротких замикань у приміщенні встановлено електричний щиток з автоматичними запобіжниками.

Таким чином, виконуються усі передбачені НПАОП~0.00-1.28-10 заходи електробезпеки.

\section{Пожежна безпека}
У приміщенні розміщені тверді та волокнисті займисті речовини, дерев’яні меблі, папір, тканини. Це дає підставу віднести його до категорії пожежонебезпечних П-ІІа та до категорії В за вибухо- та пожежонебезпекою згідно НАПБ~Б.07.005-86.

У приміщенні має бути розміщено два вуглекислотних вогнегасники ВВК-2 згідно вимог НАПБ~Б.03.001-2004.

Приміщення, враховуючи розміри та вимоги ДБН~В.1.1-7-2002, має бути обладнано одним димовим пожежним датчиком ДІП-3. Такий датчик має бути встановлений на стелі близько середини кімнати та підключений до диспетчерського пожежного пункту.

Шляхи евакуації відповідають вимогам ДНАОП~0.01-1.01-95, план евакуації розміщено біля сходових клітин поверху, що виконані з непальних матеріалів та виділені з обсягу будинку.

Таким чином, усі норми пожежної безпеки задовольняють вимоги НАПБ~А.01.001-95 та інших нормативних актів.

\conclusion


\begin{thebibliography}
\bibitem{Landafshitz}
Ландау~Л.~Д., Лившиц~Е.~М. Теоретическая физика: Учеб. пособ.: Для вузов. В 10 т. Т. VI. Гидродинамика. --- 5-е изд., стереот. --- М.:~ФИЗМАТЛИТ. --- 2001. --- 736~с. --- ISBN5-9221-0121-8 (T. VI).

\bibitem{Sokolofsky}
Sokolofsky~S.~A., Jirka~J.~A. CVEN 489-501: Special Topics on Mixing and Transport in the Environment. Engineering -- Lectures. --- 5th edition. --- Texas A \& M University. --- 2005. --- 184~pp.

\bibitem{Bejan}
Adrian~Bejan. Convection Heat Transfer. --- 4th edition. --- Wiley. --- 2013. --- 696~pp.

\bibitem{Lewis}
Lewis~R.~W. Fundamentals of the Finite Element Method for Heat and Fluid Flow. / Ronald~W.~Lewis, Perumal Nithiarasu, Kankanhally N. Seetharamu. --- Wiley. --- 2004. --- 356~pp.

\bibitem{Vlasova}
Власова~Е.~А. Приближенные методы математической физики. / Е.~А.~Власова, В.~С.~Зарубин, Г.~Н.~Кувырикин. --- М.:~Изд-во МГТУ им.~Н.~Э.~Баумана. --- 2001. --- 700~с.

\bibitem{Brebbia}
Бреббия~К. Метод граничных элементов: Пер. с англ. / Бреббия~К., Теллес~Ж., Вроубел~Л. --- М.:~Мир. --- 1987. --- 524~с.

\end{thebibliography}

\append{А}{Вихідні коди програмних засобів}


\append{Б}{Ілюстративний матеріал}


\end{document}

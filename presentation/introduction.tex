\intro
Го -- стародавня китайська стратегічна гра на двох гравців. Вона є антагоністичною грою з повною інформацією. В Го грають два гравці -- ``Чорні'' та ``Білі''. Вони по черзі розміщують на дошці, що складається з перетину 19 на 19 ліній, камені свого кольору. Го територіальна гра, тобто гравець, що під кінець гри має більшу територію -- виграє. Як і для багатьох інших подібних ігор, були спроби створити комп'ютерні програми, що гарно грають в Го, але це виявилося справжнім викликом для програмістів. Складність обчислення партій в Го на кілька порядків більша за шахи. На кожному кроці можливі близько 200—300 ходів, статична ж оцінка життя груп каменів фактично неможлива. Одним ходом тут можна цілком зіпсувати всю гру, навіть коли решта ходів були дуже добрі. Тому програми для гри в Го не використовують таких алгоритмів, як шахові програми, а замість цього зазвичай мають кілька десятків модулів для оцінки різних аспектів гри і під час аналізу намагаються використовувати ті ж самі поняттями, що й люди. Попри це вони і далі грають дуже слабко та програють навіть не дуже сильним аматорам.

% Перша програма для гри в Го була написана Альбертом Зобріст в 1968 році як частина дисертації по розпізнаванню образів. Він використав функцію впливу для оцінкі території і Зобріст-хешування для виявлення ситуацій Ко.

% Протягом довгого часу було широко поширена думка, що комп'ютерне Го являє собою проблему, що в корені відрізняється від комп'ютерних шахів, оскільки вважалося, що методи, що спираються на швидкий глобальний пошук дадуть гірший результат, порівняно з експертою оцінкою гри. Саме тому більша частина зусиль в області розвитку комп'ютерних програм того часу була зосереджена на експертних системах. Були спроби об'єднати такі системи з локальним покушом, аби дати можливість програмі відповідати на питання тактичного характеру. Результатом цього були створені програми, які мали здатність оброблювати багато різних локальних ситуацій, але які мали дуже виражені недоліки у загальній тактиці гри. Крім того, ці класичні програми не отримали майже нічого від збільшення доступних обчислювальних потужностей і прогрес в цій області в цілому йшов дуже повільно.

% Кілька дослідників зрозуміли потенціал імовірнісних методів і передбачили, що вони будуть домінувати у комп'ютерному Го. Але у той же час вони розуміли те, що гарну ігрову програму можна буде розробити тільки в далекому майбутньому, в результаті фундаментальних досягнень в загальній технології штучного інтелекту. Навіть написати програму, здатну автоматично визначати переможця готової грі було не тривіальним завданням.

% Поява програм, заснованих на пошуку Монте-Карло, кардинально змінила ситуацію, хоча розрив між професійними гравцями і найсильнішими програми все ще залишається значним. Останні розробки у цих методах пошуку по деревах і машинному навчанню принесли кращі результати для програм, що грають в Го на маленькій дошці 9x9. У 2009 році з'явилися перші подібні програми, що могли змагатися з низькими данами на дошках розміром 19х19.

Усі програми, що грають в Го повинні оперувати деякими структурами даних (наприклад такими, що репрезентують поточний стан дошки або гри). Більшість з таких програм засновані на створенні великої кількості різних партій, та подальшому їх аналізі. Саме тому питання пошуку у подібних структурах даних дуже актуальне, адже його оптимізація корінним чином вплине на роботу та якість програм, що грають в Го.
